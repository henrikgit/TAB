%==========================
%Abstandsdefinitionen
%==========================
\newlength{\xAbstand}
\setlength{\xAbstand}{4pt}
\newlength{\yAbstand}
\setlength{\yAbstand}{10pt}

%==========================
%Überschriften wie \titel, \abschnitt usw. - TikZ Variante mit Farben
%==========================
%Variante - unterstrichen, Balken links
%\renewcommand{\abschnitt}[1]{%
%%\raisebox{-0.2\height}{%
%\begin{tikzpicture}[
%remember picture,
%]
%\node[%
%%text width=\linewidth,
%%draw=\abschnitthintergrund,
%inner xsep=0pt,
%inner ysep=2pt,
%outer ysep=0pt,
%%fill=\abschnitthintergrund,
%font=\bfseries\Large,
%text=abschnitttext,
%%minimum height=0.5cm,
%]
%(AbschnittNode) {%
%#1%
%};%
%\end{tikzpicture}% <-- muss sein sonst ist Leerzeichen dabei was SEHR SCHLECHT ist
%\begin{tikzpicture}[
%remember picture,
%overlay,
%]
%\draw[
%inner sep=0pt
%]
%($(AbschnittNode.west -| current page text area.west)+(4pt,8pt)$) coordinate (AbschnittBoxRO)
%($(AbschnittNode.west -| current page text area.west)+(4pt,-8pt)$) coordinate (AbschnittBoxRU)
%($(AbschnittNode.west -| current page text area.west)+(0pt,-8pt)$) coordinate (AbschnittBoxLU)
%($(AbschnittNode.west -| current page text area.west)+(0pt,8pt)$) coordinate (AbschnittBoxLO)
%;
%\draw[ultra thick, draw=abschnittlinie, yshift=-5pt] (AbschnittNode.south -| current page text area.west) -- (AbschnittNode.south -| current page text area.east);
%\fill [draw=abschnittlinie, fill=abschnittlinie,] (AbschnittBoxLU) rectangle (AbschnittBoxRO);
%%veraltet, nicht gut
%%\filldraw[ultra thick, draw=abschnittlinie, fill=abschnittlinie,] (AbschnittNode.south -| current page text area.west) rectangle ($(AbschnittNode.north -| current page text area.west)-(5pt,0)$);
%%\draw[ultra thick, abschnittlinie] ($(AbschnittNode.west -| current page text area.west)+(0pt,-8pt)$) -- ($(AbschnittNode.west -| current page text area.east)+(0pt,-8pt)$);
%\end{tikzpicture}%
%%}
%}
%
%Variante mit Hinterlegung Hintergrundfarbe Zeile
\renewcommand{\abschnitt}[1]{%
%
%\raisebox{-0.2\height}{%
\begin{tikzpicture}[
remember picture,
overlay,
]
\node[%
%text width=\linewidth,
%draw=\abschnitthintergrund,
inner xsep=0pt,
inner ysep=2pt,
outer ysep=0pt,
%fill=\abschnitthintergrund,
font=\bfseries\Large\sffamily,
%font=\bfseries\Large\rmfamily,
%text=abschnittlinie,
text=abschnitttext,
%text=white,
anchor=south east,
%minimum height=0.5cm,
]
(AbschnittNode) {%
#1%
};%
%\end{tikzpicture}% <-- muss sein sonst ist Leerzeichen dabei was SEHR SCHLECHT ist
%\begin{tikzpicture}[
%remember picture,
%overlay,
%]
\draw[
inner sep=0pt,
outer sep=0pt,
]
($(AbschnittNode.west -| current page text area.west)+(\xAbstand,\yAbstand)$) coordinate (AbschnittBoxRO)
($(AbschnittNode.west -| current page text area.west)+(\xAbstand,-\yAbstand)$) coordinate (AbschnittBoxRU)
($(AbschnittNode.west -| current page text area.west)+(0pt,-\yAbstand)$) coordinate (AbschnittBoxLU)
($(AbschnittNode.west -| current page text area.west)+(0pt,\yAbstand)$) coordinate (AbschnittBoxLO)
;
\begin{pgfonlayer}{background} %Beginn unterlegter Text
%\fill[ultra thick, abschnittlinie] (AbschnittNode.south -| current page text area.west) rectangle (AbschnittNode.north -| current page text area.east);
\fill[%
inner sep=0pt,
outer sep=0pt,
abschnittlinie,
%abschnitttext,
] (AbschnittBoxLU) rectangle (AbschnittBoxRO.north -| current page text area.east);
\end{pgfonlayer}  %Ende unterlegter Text
%\draw[ultra thick,
%abschnittlinie,
%abschnitttext,
%] ($(AbschnittNode.west -| current page text area.west)+(0pt,\yAbstand)$) -- ($(AbschnittNode.west -| current page text area.east)+(0pt,\yAbstand)$); %passende darüberliegende Linie zu vorherigem Block
\fill[
inner sep=0pt,
outer sep=0pt,
%draw=abschnittlinie,
%fill=abschnittlinie,
%draw=abschnitttext,
fill=abschnitttext,
] (AbschnittBoxLU) rectangle (AbschnittBoxRO); %dicker Blocke links an Seitenrand
%\draw[ultra thick, abschnittlinie] ($(AbschnittNode.west -| current page text area.west)+(0pt,-\yAbstand)$) -- ($(AbschnittNode.west -| current page text area.east)+(0pt,-\yAbstand)$); %passende darunterliegende Linie zu vorherigem Block
%\draw[ultra thick, abschnittlinie] (AbschnittBoxLU) -- (AbschnittBoxLU -| current page text area.east); %passende darunterliegende Linie zu vorherigem Block
%\filldraw[ultra thick, draw=abschnittlinie, fill=abschnittlinie,] (AbschnittNode.south -| current page text area.west) rectangle ($(AbschnittNode.north -| current page text area.west)-(5pt,0)$); %großer Block links
\end{tikzpicture}%
%}
}
%
%
\newcommand{\kurzprofilteildesign}[1]{%
\draw
($({#1}.west -| current page text area.west)+(-4pt,8pt)$) coordinate (KurzprofilteilBoxRO)
($({#1}.west -| current page text area.west)+(-4pt,-8pt)$) coordinate (KurzprofilteilBoxRU)
($({#1}.west -| current page text area.west)+(0pt,-8pt)$) coordinate (KurzprofilteilBoxLU)
($({#1}.west -| current page text area.west)+(0pt,8pt)$) coordinate (KurzprofilteilBoxLO)
;
%\fill [draw=abschnittlinie, fill=abschnittlinie,] (KurzprofilteilBoxLU) rectangle (KurzprofilteilBoxRO);
}

%==========================
%Hintergrund
%==========================
%\usepackage{background}
%\newcommand{\MyTikzBackground}{% For a logo drawn with TikZ
%\begin{tikzpicture}[
%remember picture,
%overlay,
%draw=black,
%ultra thick
%]
%\fill[
%abschnitttext,
%]
%([xshift=-10pt,yshift=10pt]current page text area.north west) rectangle ([xshift=10pt,yshift=-10pt]current page text area.south east);
%\fill[
%abschnittlinie,
%]
%([xshift=-5pt,yshift=5pt]current page text area.north west) rectangle ([xshift=5pt,yshift=-5pt]current page text area.south east);
%\fill[
%white,
%]
%(current page text area.north west) rectangle (current page text area.south east);
%%\fill ()
%\end{tikzpicture}
%}
%
%\SetBgContents{\MyTikzBackground}% Set tikz picture
%
%\SetBgPosition{current page.north west}% Select location
%\SetBgOpacity{1.0}% Select opacity
%\SetBgAngle{0.0}% Select roation of logo
%\SetBgScale{1.0}% Select scale factor of logo
\usepackage[T1]{fontenc}
\usepackage[utf8]{inputenc}

%========================
%Variante Helvetica + Times New Roman
%========================

\usepackage{amsmath}
\usepackage{newtxtext,newtxmath}
\usepackage[scaled=0.9]{beramono}
\usepackage{sansmath}

\usepackage{microtype}
\LoadMicrotypeFile{ptm} %lieber mal die eingebaute Microtype-Konfiguration nehmen, sieht besser aus und ist vor allen Dingen ausführlicher

%========================
%Variante Bera + Open Sans
%========================

%\usepackage{amsmath,amssymb}
%\usepackage{bera}
%\usepackage[defaultsans,scale=0.95]{opensans}
%
%\usepackage{microtype}

%========================
%Variante Helvetica + Palatino
%========================

%\usepackage{amsmath}
%\usepackage{sansmath}
%\usepackage{tgheros}
%\usepackage[scaled=0.9]{beramono}
%\usepackage{newpxtext,newpxmath}
%
%\usepackage{microtype}
%%\LoadMicrotypeFile{ppl}

%========================
%Variante Helvetica + Bitstream Charter
%========================

%\usepackage{amsmath}
%\usepackage{sansmath}
%%
%%\usepackage[bitstream-charter,expert]{mathdesign}
%%%\usepackage{XCharter}
%%\usepackage[sups]{XCharter}
%%\usepackage[scale=0.9142]{tgheros}
%%\usepackage[scaled=0.9]{beramono}
%%
%\usepackage[bitstream-charter,expert]{mathdesign}
%%\usepackage{XCharter}
%\usepackage[sups]{XCharter}
%\usepackage[scale=0.95]{tgheros}
%\usepackage[scaled=0.9]{beramono}
%%%
%\usepackage{microtype}

%%folgendes nur getestet, finde Voreinstellung schicker
%\usepackage{microtype}
%\microtypesetup{
%	babel=true,
%	protrusion=true,
%	expansion,
%	activate={true,nocompatibility}, % activate={true,nocompatibility} - activate protrusion and expansion
%	final, % final - enable microtype; use "draft" to disable
%	%draft,
%	tracking=true, % tracking=true, kerning=true, spacing=true - activate these techniques
%	kerning=true,
%	spacing=true,
%	factor=1100, % factor=1100 -> adds 100 = 10% to the protrusion amount (default is 1000)
%	stretch=10, shrink=10 % stretch=10, shrink=10 - reduce stretchability/shrinkability (default is 20/20)
%}
%\SetTracking{encoding={*}, shape=sc}{15}

%Quelle: http://tex.stackexchange.com/questions/135062
%\sisetup{
%	math-micro=\muup,
%	math-ohm=\Omegaup,
%	text-micro={\fontfamily{mdbch}\textmu},
%	text-ohm={\fontfamily{mdbch}\textohm}
%}

%========================
%Variante Baskerville + Kepler Math
%========================

%%\usepackage[
%%%nosf,
%%notextcomp,
%%]{kpfonts}
%\usepackage{tgheros}
%\usepackage{baskervald}
%\usepackage{amsmath}
%\usepackage{sansmath}

%========================
%Variante Garamond
%========================

%\usepackage{amsmath}
%\usepackage[urw-garamond]{mathdesign}
%\usepackage[scale=0.95]{tgheros}
%%%\usepackage[default]{gfsneohellenic}
%%%\usepackage[LGR,T1]{fontenc}
%\usepackage{sansmath}

%========================
%Variante Palatino + Avant Garde mit Palatino Mathe
%========================

%\usepackage{amsmath}
%\usepackage{avant}
%\usepackage{tgpagella}
%\usepackage{newpxtext} %Palatino + Helvetica
%\usepackage[scaled=0.92]{beramono} %falls genutzt, \small im Befehl \tart entfernen! - wirkt aber doch zu stark, und schon gar nicht unskaliert nutzen!
%\usepackage{bookman}
%\usepackage{sansmath}



%========================
%Variante Latin Modern
%========================

%\usepackage{amsmath}
%\usepackage{lmodern}
%\usepackage{sansmath}

%========================
%Variante Linux Libertine
%========================

%\usepackage{amsmath}
%\usepackage{sansmath}
%\usepackage{libertine}
%%\usepackage[scale=0.9]{tgheros}
%\usepackage[libertine,varg,cmintegrals,cmbraces]{newtxmath}
%\usepackage[scaled=0.8]{beramono}


%%%\let\iint\undefined
%%%\let\iiint\undefined
%%%\let\iiiint\undefined
%%%\let\idotsint\undefined
%%\usepackage{amsmath}
%%\usepackage{sansmath}
%%%After loading math package, switch to osf in text, for which the following 4 lines are required. This is not so obvious.
%%\makeatletter
%%\def\libertine@figurestyle{OsF} %affects \libertine macro
%%\makeatother
%%\renewcommand*{\rmdefault}{LinuxLibertineT-OsF} % for normal text

%========================
%Variante Minion Pro + Myriad Pro
%========================

%\usepackage{MinionPro}
%\usepackage{MyriadPro}
%\usepackage[scaled=0.8]{beramono}
%
%\usepackage{microtype}
%\LoadMicrotypeFile{pmnx}

%========================
%Variante Kepler Fonts
%========================

%\usepackage{amsmath}
%\usepackage[
%%nomath,
%%notext,
%notextcomp,
%%noamsmath
%]{kpfonts}
%\usepackage{tgheros}
%\usepackage[scaled=0.9]{beramono}
%\usepackage{sansmath}

%%========================
%%Variante X
%%========================
%
%\usepackage{amsmath,amssymb}
%\usepackage{bera}
%\usepackage[defaultsans,scale=0.95]{opensans}
%%\usepackage{sourcesanspro}



\renewcommand{\familydefault}{\sfdefault}
%\renewcommand{\sfdefault}{\rmdefault}
%\addtokomafont{part}{\mdseries}
%\addtokomafont{partnumber}{\mdseries}










%\input{Einstellungen/glyphtounicode}
%\pdfgentounicode=1
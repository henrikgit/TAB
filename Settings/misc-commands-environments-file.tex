%========================
%bla Finances.ods bla
%========================
\newcommand{\sheetname}[1]{\textsf{#1}\xspace}
\newcommand{\loc}{LibreOffice~Calc\xspace}
\newcommand{\tfn}{Finances.ods\xspace}
\newcommand{\structurenext}{\( \blacktriangleright \)\xspace}
%\newenvironment{specialnote}{\ignorespaces\begin{center}\ignorespaces\begin{minipage}{0.75\linewidth}\sffamily\singlespacing}{\end{minipage}\end{center}\ignorespacesafterend}
%\newenvironment{specialnote}{\ignorespaces\par\centering\begin{minipage}{0.8\linewidth}\singlespacing\small\sffamily}{\end{minipage}\par}
\newenvironment{specialnote}{\vspace{2ex}\par\centering\begin{minipage}{0.8\linewidth}\small\sffamily}{\end{minipage}\par\vspace{2ex}}

%========================
%um das Leben im Umgang mit LaTeX einfacher zu machen
%========================

%allgemeine Befehle zum Schreiben
\newcommand{\system}[1]{\textit{#1}\xspace}
\newcommand{\sterm}[1]{\textit{#1}\xspace}
\newcommand{\programmcode}[1]{\texttt{#1}\xspace}
\newcommand{\myfont}[1]{\fontfamily{#1}\selectfont}
%\newcommand{\fullref}[1]{\autoref{#1} \nameref{#1}}
\newcommand{\fullref}[1]{\autoref{#1}, \autopageref{#1}}


%Fix für subscripts zu tief wenn superscript dabei ist Quelle: http://tex.stackexchange.com/a/5192/25683
%wird in den nachfolgenden Makros verwendet, muss daher hierhin/an den Anfang
\makeatletter
\newcommand{\raisemath}[1]{\mathpalette{\raisem@th{#1}}}
\newcommand{\raisem@th}[3]{\raisebox{#1}{$#2#3$}}
\makeatother

%Fix für Differential
%http://tex.stackexchange.com/a/60546/25683
\newcommand*\diff{\mathop{}\!\mathrm{d}}
\newcommand*\Diff[1]{\mathop{}\!\mathrm{d^#1}}

%Abkürzungen
\newcommand{\ie}{i.\,e.\xspace}
\newcommand{\Ie}{I.\,e.\xspace}
\newcommand{\eg}{e.\,g.\xspace}
\newcommand{\Eg}{E.\,g.\xspace}
\newcommand{\zB}{z.\,B.\xspace}
\newcommand{\uA}{u.\,A.\xspace}
\newcommand{\idR}{i.\,d.\,R.\xspace}
\newcommand{\etal}{et al.\xspace}
\newcommand{\twod}{2\textsc{d}\xspace}
\newcommand{\threed}{3\textsc{d}\xspace}

%Bildereinbindung
\newcommand{\bildzeile}[1]{\raisebox{-.1\height}{\includegraphics[height=0.9\baselineskip]{#1}}}
\newcommand{\begriffbildzeile}[2]{\emph{#1}~\raisebox{-.1\height}{\includegraphics[height=0.9\baselineskip]{#2}}}

\newlength{\abcd}

%eigene Umgebung auf Basis des comment-Pakets
%noch nichts

%========================
%SI units stuff
%========================
%\DeclareSIUnit[options]{unit}{symbol}
\newcommand{\money}[2]{\SI{#1}{#2}}


%========================
%TikZ
%========================
%Text durchstreichen
\newcommand{\tcancel}[2][black]{%
	\begin{tikzpicture}[baseline={(a.base)}]
	\node[draw=#1, thick, cross out, inner sep=0pt, outer sep=0pt] (a){#2};
	\end{tikzpicture}%
}
%Text in Box
\newcommand{\tbox}[1]{\raisebox{-0.225\height}{\tikz \node[draw, solid, inner sep=2pt, outer sep=3pt, font=\ttfamily\small] {#1};}\xspace}
%Keyboard Stroke Tastatur Tastendruck-Kasten mit TikZ
\newcommand{\keystroke}[1]{%
	\tikz[
	baseline=(key.base),
	]
	\node[%
	draw,
	fill=white!10,
	drop shadow={
		shadow xshift=1pt,
		shadow yshift=-1pt,
		fill=black,
		opacity=0.75
	},
	rectangle,
%	rounded corners=0.5pt,
%	text height=1.4ex,
	text depth=0.3ex,
	inner ysep=1pt,
	outer ysep=0pt,
	inner xsep=2pt,
	line width=0.5pt,
	font=\sffamily\small,
	] (key) {#1\strut}
	;
}
%\newcommand{\folderpath}[#1]{}
%\newcommand{\menubuttons}[#1]{\menu{#1}}

%================================
%Mathe
%================================
\newcommand{\suli}[2]{\sum\limits_{#1}^{#2}}
\newcommand{\ri}{\longrightarrow}
\newcommand{\intli}[2]{\int\limits_{#1}^{#2}}
\newcommand{\proli}[2]{\prod\limits_{#1}^{#2}}
\newcommand{\dunderline}[1]{\underline{\underline{#1}}}
\newcommand{\uunder}[1]{\underline{\underline{#1}}}
\newcommand{\entspricht}{\mathrel{\hat{=}}} 
\newcommand{\defals}{\mathrel{\mathop:}=}
\newcommand{\grad}{\ensuremath{^\circ}}
\newcommand{\und}{\wedge}
\newcommand{\abs}[1]{\lvert{#1}\rvert}
\newcommand{\norm}[1]{\lVert{#1}\rVert}

%================================
%pgfplots
%================================
\newcommand{\getpgfkey}[1]{\pgfkeysvalueof{/pgfplots/#1}}
\newcommand{\xmin}[0]{\getpgfkey{xmin}}
\newcommand{\xmax}[0]{\getpgfkey{xmax}}
\newcommand{\ymin}[0]{\getpgfkey{ymin}}
\newcommand{\ymax}[0]{\getpgfkey{ymax}}

%========================
%für Paket todonotes
%========================
\newcommand{\ausarbeiten}[1]{\todo[color=green!50]{{#1} ausarbeiten}}
\newcommand{\refrein}[1]{\todo[color=red]{#1}}

%========================
%eigene Farben
%========================
\colorlet{mygray}{black!65}

%Weg
\newcommand{\Wegmin}{s^{\text{min}}}

%Kraft Kräfte
\newcommand{\Fx}{F_{x}}
\newcommand{\Fy}{F_{y}}
\newcommand{\Fz}{F_{z}}
\newcommand{\Fxmin}{F_{\raisemath{2pt}{x}}^{\text{min}}}
\newcommand{\Fymin}{F_{\raisemath{2pt}{y}}^{\text{min}}}
\newcommand{\Fzmin}{F_{\raisemath{2pt}{z}}^{\text{min}}}
\newcommand{\Fxmax}{F_{\raisemath{2pt}{x}}^{\text{max}}}
\newcommand{\Fymax}{F_{\raisemath{2pt}{y}}^{\text{max}}}
\newcommand{\Fzmax}{F_{\raisemath{2pt}{z}}^{\text{max}}}

%Zugversuch
\newcommand{\dL}{\diff L}

%Winkel
\newcommand{\WinkelFuenfundvierzig}{\SI{45}{\degree}\xspace}
\newcommand{\rechterWinkel}{\SI{90}{\degree}\xspace}

%Einheiten
\newcommand{\Mikrometer}{\si{\micro\metre}\xspace}
\newcommand{\Millimeter}{\si{\milli\metre}\xspace}

\newcommand{\MATRIXAngeben}[2]{Matrix \( {#1} \times {#2} \)}
%
%Weg
\newcommand{\WegZehnMM}{\SI{10}{\milli\metre}\xspace}

%========================
%eigene Zähler
%========================
\newcounter{asdfu}
\setcounter{asdfu}{1}
\newcommand{\mycount}{\arabic{asdfu}\addtocounter{asdfu}{1}.} %erst die Ausgabe des Zählerstands, dann Erhöhung um 1

\newcounter{Marke}
\setcounter{Marke}{1}
\newcommand{\BeschrCount}{\arabic{Marke}\addtocounter{Marke}{1}} %erst die Ausgabe des Zählerstands, dann Erhöhung um 1

\newcounter{TextteilCount}
\setcounter{TextteilCount}{0}
\newcommand{\textteilzaehler}{\addtocounter{TextteilCount}{1}\arabic{TextteilCount}}
%Zähler TextteilCount sollen pro \subsubsection zurückgesetzt werden!
\makeatletter
\@addtoreset{TextteilCount}{subsubsection}
\makeatother

%========================
%checkmark definieren wg. komischen symbol-font-problemen
%========================
\usepackage{pifont}
\newcommand{\ja}{\ding{51}\xspace}%
\newcommand{\nein}{\ding{53}\xspace}
%========================
%Variante Helvetica + Bitstream Charter
%========================
%\usepackage[bitstream-charter,expert]{mathdesign}
%\expandafter\let\csname T1+fvm\endcsname\relax
%\usepackage[sups]{XCharter}
%\usepackage[scaled=0.935]{beramono}
%\usepackage[scale=0.95]{tgheros}
%\usepackage{microtype}
%\setstretch{1.075}

%========================
%Variante Helvetica + Palatino
%========================
%\usepackage{newpxtext,newpxmath}
%\usepackage[scaled=0.885]{beramono}
%\usepackage{microtype}
%\setstretch{1.05}

%========================
%Variante Helvetica + Linux Libertine
%========================
%\usepackage{amsmath}
%\usepackage{libertine}
%\usepackage{libertinust1math}
%\usepackage[scaled=0.825]{beramono}
%\usepackage[scale=0.88]{tgheros}
%\usepackage{microtype}

%========================
%Variante Helvetica + Times New Roman
%========================
%\usepackage{amsmath}
%\usepackage{newtxtext,newtxmath}
%\usepackage[scaled=0.9]{beramono}
%\usepackage{sansmath}
%\usepackage{microtype}
%\LoadMicrotypeFile{ptm} %lieber mal die eingebaute Microtype-Konfiguration nehmen, sieht besser aus und ist vor allen Dingen ausführlicher

%========================
%Variante Minion Pro + Myriad Pro
%========================
%\usepackage{MinionPro}
%\usepackage[lf]{MyriadPro}
%\usepackage[scaled=0.9]{beramono}
%\usepackage{microtype}
%%\LoadMicrotypeFile{pmnx}
%\input{glyphtounicode}
%\pdfgentounicode=1
%%%\setstretch{1.075}!!

%========================
%Variante Roboto + Linux Libertine
%========================
%\usepackage{amsmath}
%\usepackage{libertine}
%\usepackage{libertinust1math}
%\usepackage[
%%sfdefault,
%%light,
%scale=0.92
%]{roboto}
%\usepackage{inconsolata}
%\usepackage{microtype}
%%### alle Überschriften in normaler (nicht-fetter) Schrift setzen
%\addtokomafont{disposition}{\mdseries}

%========================
%Variante Roboto + Bitstream Charter
%========================
%\usepackage[
%sfdefault,
%light,
%scale=0.925
%]{roboto}
%\usepackage[
%%scaled=0.98
%]{beramono}
%\usepackage[
%bitstream-charter,
%expert
%]{mathdesign}
%\usepackage[sups]{XCharter}
%\usepackage{microtype}

%========================
%Variante Gillius + Latin Modern
%========================
%%%%\usepackage{tgschola}
%\usepackage{lmodern}
%\usepackage[default]{gillius}
%\usepackage[scaled=0.9]{beramono}
%\usepackage{microtype}
%\setlength{\minipagetaetbreite}{13.75cm}
%\renewcommand{\familydefault}{\sfdefault}

%========================
%Variante Carlito
%========================
%\usepackage[sfdefault,lf]{carlito}
%%% The 'lf' option for lining figures
%%% The 'sfdefault' option to make the base font sans serif
%\renewcommand*\oldstylenums[1]{\carlitoOsF #1}
%\usepackage{microtype}
%\renewcommand{\familydefault}{\sfdefault}

%========================
%Variante Bera + Open Sans
%========================
%\usepackage{amsmath,amssymb}
%\usepackage{bera}
%\usepackage[defaultsans,scale=0.95]{opensans}
%\usepackage{microtype}
%\renewcommand{\familydefault}{\sfdefault}

%========================
%Variante Baskerville + Kepler Math
%========================
%%\usepackage[
%%%nosf,
%%notextcomp,
%%]{kpfonts}
%\usepackage{tgheros}
%\usepackage{baskervald}
%\usepackage{amsmath}
%\usepackage{sansmath}

%========================
%Variante Garamond
%========================
\usepackage{amsmath}
\usepackage[urw-garamond]{mathdesign}
\expandafter\let\csname T1+fvm\endcsname\relax
\renewcommand\bfdefault{bx}
\usepackage[scaled=1.05]{zlmtt}
\usepackage[scale=1.025]{gillius}
\usepackage{sansmath}
\usepackage{microtype}
%\renewcommand{\familydefault}{\sfdefault}

%========================
%Variante Palatino + Avant Garde mit Palatino Mathe
%========================
%\usepackage{amsmath}
%\usepackage{avant}
%%%\usepackage{newpxtext} %Palatino + Helvetica
%\usepackage{bookman}
%\usepackage{sansmath}
%\renewcommand{\familydefault}{\sfdefault}

%========================
%Variante Latin Modern
%========================
%\usepackage{amsmath}
%\usepackage{lmodern}
%\usepackage{sansmath}

%========================
%Variante Helvetica + Linux Libertine
%========================
%\usepackage{amsmath}
%\usepackage{libertine}
%\usepackage[scale=0.95]{tgheros}
%\usepackage[scaled=0.83]{beramono}
%\renewcommand{\familydefault}{\sfdefault}
%%%\usepackage[libertine,varg,cmintegrals,cmbraces]{newtxmath}


%%%\let\iint\undefined
%%%\let\iiint\undefined
%%%\let\iiiint\undefined
%%%\let\idotsint\undefined
%%\usepackage{amsmath}
%%\usepackage{sansmath}
%%%After loading math package, switch to osf in text, for which the following 4 lines are required. This is not so obvious.
%%\makeatletter
%%\def\libertine@figurestyle{OsF} %affects \libertine macro
%%\makeatother
%%\renewcommand*{\rmdefault}{LinuxLibertineT-OsF} % for normal text

%========================
%Variante Kepler Fonts
%========================

%\usepackage{amsmath}
%\usepackage[
%light,
%%nomath,
%%notext,
%sfmath,
%notextcomp,
%%noamsmath
%]{kpfonts}
%%\usepackage{tgheros}
%\usepackage[scaled=0.9]{beramono}
%%\usepackage{sansmath}
%%\usepackage{lato}
%\usepackage{tgadventor}
%\usepackage{microtype}
%%\renewcommand{\familydefault}{\sfdefault}

%
%\usepackage[defaultfam,light,tabular,lining]{montserrat} %% Option 'defaultfam'
%%% only if the base font of the document is to be sans serig
%\usepackage[T1]{fontenc}
%\renewcommand*\oldstylenums[1]{{\fontfamily{Montserrat-TOsF}\selectfont #1}}
%\usepackage{microtype}

%%========================
%%Variante Chivo
%%========================

%\usepackage[familydefault,light]{Chivo} %% Option 'familydefault' only if the base font of the document is to be sans serif
%\setlength{\minipagetaetbreite}{13cm}
%\usepackage[
%	spacing=true,
%	protrusion=true,
%	kerning=true,
%	letterspace=-350, % factor=1100 -> adds 100 = 10% to the protrusion amount (default is 1000)
%]{microtype}
%\renewcommand{\familydefault}{\sfdefault}


%========================
%Variante Source Sans 
%========================
%\usepackage{amsmath,amssymb}
%\usepackage{sansmath}
%\usepackage[]{sourceserifpro}
%\usepackage[light]{sourcesanspro}
%\usepackage{microtype}
%\setlength{\minipagetaetbreite}{13.7cm}
%\renewcommand{\familydefault}{\sfdefault}

%\renewcommand{\familydefault}{\sfdefault}
%\renewcommand{\sfdefault}{\rmdefault}
%\addtokomafont{part}{\mdseries}
%\addtokomafont{partnumber}{\mdseries}

\usepackage[T1]{fontenc}
\usepackage[utf8]{inputenc}

\input glyphtounicode
\pdfgentounicode=1

%%%Passende Fontgrößen herausfinden:
%\begin{itemize}
%	\item A in \sbox0{A}Serif: \the\ht0
%	\item A in \sbox0{\sffamily A}Sans: \the\ht0
%	\item A in \sbox0{\ttfamily A}Mono: \the\ht0
%\end{itemize}
\usepackage[english]{babel}

\usepackage[
%x11names,
svgnames,
]{xcolor}

\usepackage[
%draft
]{graphicx}
\usepackage{
%textcomp, % auskommentieren falls newtxtext geladen
eso-pic,
multicol,
longtable,
booktabs,
setspace,
xcolor,
lastpage,
hhline,
xspace,
enumitem,
colortbl,
%wasysym, %für \phone Symbol
%marvosym, %für \mobilefone Symbol
comment,
pdflscape,
listings,
adjustbox,
}

\usepackage[
textsize=footnotesize,
]{todonotes}

\usepackage{acro}
\acsetup{
short-format = {\scshape},
}

\usepackage[
labelfont=sf,
%%%%labelfont=sc, %Kapitälchen, passt nicht wg. nicht-osf Ziffern
%%%%labelfont=it, %italics, nicht so gutaussehend
%%%labelfont=sl, %slanted, nicht so gutaussehend
hypcap=false,
format=hang,
%margin={2cm,2cm},
%width=0.8\columnwidth,
width=0.75\textwidth,
]{caption}

%============================
%SI Einheiten Paket siunitx
%============================
\usepackage[
%per-mode=fraction,
%locale=DE,
locale=US,
%math-sf,
text-sf,
%detect-all, %For convenience, all of the preceding options can be turned on or off in one go using the detect-all and detect-none. As the names indicate, detect-all sets all of detect-weight, detect-family, detect-shape and detect-mode to true.
detect-mode=true, %the current mode (text or math) is detected using the detect-mode switch
separate-uncertainty=true,
retain-explicit-plus,
%binary-units,
%range-phrase = {\text{--}},
%range-phrase = {~\text{bis}~},
%group-separator={\,}, %nur für Deutsche- bzw. Euro-Version
%Tabellenformatierung für S-Spalten
%immer lokal machen!!!!
]{siunitx}

%\setlist[itemize]{noitemsep, nosep, leftmargin=1em}

% Listen kompakter via enumitem-Paket
%\setlist{noitemsep}
%\setlist[1]{\labelindent=\parindent} % < Usually a good idea
%\setitemize{noitemsep,nosep,labelindent=1em,leftmargin=1em}
%\setenumerate{noitemsep,nosep,labelindent=\parindent}
\setitemize{noitemsep,nosep}
\setenumerate{noitemsep,nosep}
%\setlist[itemize,enumerate,2]{labelindent=\parindent}
%\setlist[itemize,enumerate,3]{labelindent=\parindent}

%\setlist[itemize,1]{label=\textbullet,leftmargin=1.2em}
%\setlist[itemize,2]{label=\textopenbullet,leftmargin=1.2em}
%\setlist[itemize,1]{label=$ \bullet $,leftmargin=1.2em}
%\setlist[itemize,2]{label=$ \circ $,leftmargin=1.2em}
\setlist[itemize,1]{label=$ \bullet $}
\setlist[itemize,2]{label=$ \circ $}

%Aufzählen 1. Ebene mit Klammer - AUSGESCHALTET
%\renewcommand{\theenumi}{\arabic{enumi}}
%\renewcommand{\labelenumi}{\theenumi{)}}
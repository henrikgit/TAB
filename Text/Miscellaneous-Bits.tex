\section{Miscellaneous Bits}
\label{sec:miscellaneous-bits}
% !TeX spellcheck = en_US

\subsection{Prepare the next year}
\label{subsec:prepare-the-next-year}

\begin{itemize}
	\item Come December 31, the very last TODO-point on your list should be to write down the last pieces of information in regards to the last day of the year.
	So literally write down \programmcode{2016-12-31} into any cell that prints today's date, \eg in the \sheetname{Budget} sheet.
	\item Otherwise the next time you open that file later than December 31 of that year, the cell that prints today's date will obviously the date of that day.
	Hence January of the next year or the years to come perhaps.
	\item Proceed in an analogical fashion for every cell that prints dates through formulas.
	\item Otherwise all the budget-sheets which feature formulas based on the current month and/or date are faulty.
	Or rather, they do not show values which make sense.
	\item 
\end{itemize}

\subsection{Forecasting}
\label{subsec:forecasting}

There is one other feature that I had originally planned to integrate: \emph{forecasting}.
Forecasting is the process of making predictions about future developments based on past and present data and most commonly by analysis of trends.\footnote{Cp. \href{https://en.wikipedia.org/wiki/Forecasting}{https://en.wikipedia.org/wiki/Forecasting}.}

I am not opposed to integrate more sheets and calculations into \tfn, but so far I have not used trends and honestly, was able to deduce the good-enough optimizations of budgeted amounts by myself.
And as stated multiple times throughout this guide, the file spans over 12 months (only), which obviously makes it impossible to form average values which account for values spanning over a longer time period than 12 months.
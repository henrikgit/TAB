\subsection{Examples for Budgeting Expenses}
\label{subsec:examples-budgeting-expenses}
% !TeX spellcheck = en_US

This subsection includes examples for showcasing how to use the budgeting sheets for expenses.
If you read through \autoref{subsec:budgeting-in-finances.ods} to \autoref{subsec:thoughts-on-budgeting}, lots of things and bits should be familiar and hopefully get cleared up if they are still a bit nebulous.

\subsubsection{Winter Coat}
\label{subsubsec:example-budgeting-expense-winter-coat}

For the example of budgeting a winter coat, let us take these facts \& figures into accounts:
\begin{itemize}
	\item We want to budget for the purchase of a super awesome winter coat which could cost up to with a value of 800\,€.
	\item It is March and it possible to buy the coat in October.
	Then the stores definitely have the new merchandise on the shelves.
\end{itemize}
The first columns in the row for that budget item could look like shown in \autoref{tab:example-budgeting-expense-winter-coat-dataentry}.
These entries yield the values shown in the gray columns in \autoref{tab:example-budgeting-expense-winter-coat-graycolumns}.\footnote{Of course the second table is directly to the right of the first one in the spreadsheet file.
I split them up to avoid for the sake of readability.}

\begin{table}[htbp]
	\centering
	\addtolength{\leftskip} {-1.25cm}
	\sffamily
	\caption[Example for Budgeting a Winter Coat: Data Entry]{Example for Budgeting a Winter Coat.
	The values from the bullet list above are displayed in the corresponding columns.}
	\label{tab:example-budgeting-expense-winter-coat-dataentry}
	\begin{tabular}{|c|l|c|c|c|c|c|c|l|}
		\hline
		\footnotesize\textbf{No.} & \rotatebox{0}{\footnotesize\textbf{Descr.}} & \rotatebox{0}{\footnotesize\textbf{Amount}} & \rotatebox{0}{\footnotesize\textbf{Start Date}} & \rotatebox{0}{\footnotesize\textbf{End Date}} & \rotatebox{0}{\footnotesize\textbf{Monthly Exp.}} & \rotatebox{0}{\footnotesize\textbf{Sum Prohib.}} & \rotatebox{0}{\footnotesize\textbf{Done}} & \rotatebox{0}{\footnotesize\textbf{Category}}\\ 
		\hline
		\hline
		1 & Winter Coat & 800 & 2019-03 & 2019-10 & 100 & No & No & Clothing/Acc.\\
		\hline
	\end{tabular}
\end{table}

\begin{table}[htbp]
	\centering
	\sffamily
	\caption[Example for Budgeting a Winter Coat: Gray Columns]{Example for Budgeting a Winter Coat.
	The values from the bullet list above are displayed in the corresponding columns.}
	\label{tab:example-budgeting-expense-winter-coat-graycolumns}
	\begin{tabular}{|c|c|c|c|c|c|}
		\hline
		\footnotesize\textbf{Cat.-No.} & \footnotesize\textbf{Months tot.} & \footnotesize\textbf{€/Month} & \footnotesize\textbf{Months in 2019} & \footnotesize\textbf{€ in 2019} & \footnotesize\textbf{Budgeted}\\
		\hline
		\hline
		\rowcolor{lightgray}
		001 & 8 & 100 & 8 & 800 & No\\
		\hline
	\end{tabular}
\end{table}

Now with \autoref{tab:example-budgeting-expense-winter-coat-graycolumns} in mind, let us circle back to \autoref{subsubsec:budgeting-item-column-sum-prohibition} and talk about its consequences for budgeting the winter coat.
As we have just read, we put 100\,€ into the virtual envelope\footnote{See \autoref{subsubsec:budgeting-the-envelope-method}.} per month.
This whole concept is displayed in \autoref{tab:bic-sum-prohibition-clarification}.\todo{ausarbeiten und mal sum prohibition aufzeigen?}

\begin{table}[htp]
	\centering
	\libertineTabular
	\caption[Example for Sum Prohibition]{An Example for Sum Prohibition.
	With each month, 100\,€ get put into the envelope and its worth finally amounts to 800\,€ in 2019-10.}
	\label{tab:bic-sum-prohibition-clarification}
	\begin{tabular}{cccc}
		\toprule
		\begin{minipage}{1.0cm}
		Nr. of\\
		Month	
		\end{minipage}
		& \begin{minipage}{1.4cm}
			Calendar\\
		 	Month	
		 \end{minipage}
		&
		\begin{minipage}{2.1cm}
			Amount going\\
			to envelope	
		\end{minipage}
		 & 
		\begin{minipage}{1.8cm}
			Total worth\\
			in envelope	
		\end{minipage}\\
%		\cmidrule(l{5.5pt}r{5pt}){2-2}
%		\cmidrule(r){1-1}
%		\cmidrule(lr){2-2}
%		\cmidrule(lr){3-3}		
%		\cmidrule(lr){4-4}
		\midrule
		1 & 2019-03 & 100 & 100\\
		2 & 2019-04 & 100 & 200\\
		3 & 2019-05 & 100 & 300\\
		4 & 2019-06 & 100 & 400\\
		5 & 2019-07 & 100 & 500\\
		6 & 2019-08 & 100 & 600\\
		7 & 2019-09 & 100 & 700\\
		8 & 2019-10 & 100 & 800\\
		\bottomrule
	\end{tabular}
\end{table}

\subsubsection{Utility: Water}
\label{subsubsec:example-budgeting-utility-water}
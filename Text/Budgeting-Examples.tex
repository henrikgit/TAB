\subsection{Examples for Budgeting Expenses}
\label{subsec:examples-budgeting-expenses}
% !TeX spellcheck = en_US

This subsection includes examples for showcasing how to use the budgeting sheets for expenses.
If you read through \autoref{subsec:budgeting-in-finances.ods-general-principle} to \autoref{subsec:advanced-tips-for-budgeting}, lots of things and bits should be familiar and hopefully get cleared up if they are still a bit nebulous.

\subsubsection{Winter Coat}
\label{subsubsec:example-budgeting-winter-coat}

For the example of budgeting a winter coat, let us take these facts \& figures into accounts:
\begin{itemize}
	\item We want to budget for the purchase of a super awesome winter coat which could cost (up to) 540\,€.
	\item It is March and it is out plan to buy the coat in October.
	Then the stores definitely have the new merchandise on the shelves.
\end{itemize}

The information given in the bullet list would lead to the columns for data entry in \autoref{tab:example-budgeting-winter-coat}, which displays how one should budget for a winter coat in \tfn.

\begin{table}[hbtp]
	\centering
	\addtolength{\leftskip}{-3.5cm}
	\setlength{\tabcolsep}{3pt}
	\sffamily
	\caption[An Example for Budgeting a Winter Coat]{An Example for Budgeting a Winter Coat.
	The cells on the left, with white background, represent the part that is for data entry.
	The values from the bullet list above are displayed in the corresponding columns.
	The last of the white columns is for the tracking category and as stated numerous times can look different in your case, if you wish to name your tracking categories differently.
	The columns for data entry yield entries in the gray columns.}
	\label{tab:example-budgeting-winter-coat}
	\begin{tabular}{|c|l|l|l|l|c|c|c|r|c|r|c|r|r|c|}
		\hline
		\begin{minipage}{0.5cm}\footnotesize\bfseries
			No.
		\end{minipage} &
		\begin{minipage}{1.5cm}\footnotesize\bfseries
			Descr.
		\end{minipage} &
		\begin{minipage}[b]{1.0cm}	\footnotesize\bfseries
			Amount
		\end{minipage} &
		\begin{minipage}[b][0.8cm]{0.8cm}\footnotesize\bfseries
			Start\\
			Date
		\end{minipage} &
		\begin{minipage}[b]{0.8cm}\footnotesize\bfseries
			End\\
			Date
		\end{minipage} &
		\begin{minipage}[b]{1.1cm}\footnotesize\bfseries
			Monthly\\
			Expense
		\end{minipage} &
		\begin{minipage}[b]{1.0cm}\footnotesize\bfseries
			Sum\\
			Prohib.
		\end{minipage} &
		\begin{minipage}[b]{0.8cm}\footnotesize\bfseries
			Done
		\end{minipage} &
		\begin{minipage}[b]{1.5cm}\footnotesize\bfseries
			Category
		\end{minipage} & %%%hier gehts los
		\footnotesize\textbf{Cat.-No.} &
		\begin{minipage}[b][0.8cm]{1cm}\footnotesize\bfseries
			Months\\
			total
		\end{minipage} &
		\footnotesize\bfseries €/Month &
		\begin{minipage}[b]{1.2cm}\footnotesize\bfseries
			Months in\\
			2019
		\end{minipage} &
		\begin{minipage}[b]{0.6cm}\footnotesize\bfseries
			€ in\\
			2019
		\end{minipage} &
		\footnotesize\bfseries Budgeted\\ 
		\hline
		\hline
		1 & Winter Coat & 540 & 2019-03 & 2019-10 & No & No & No & Clothing/Acc. & \cellcolor{lightgray}001 & \cellcolor{lightgray}8 & \cellcolor{lightgray}67.5 & \cellcolor{lightgray}8 & \cellcolor{lightgray}540.00 & \cellcolor{lightgray}No\\
		\hline
	\end{tabular}
\end{table}

The entries in white fields starting on the left in \autoref{tab:example-budgeting-winter-coat} yield the values shown in the gray columns to the right in \autoref{tab:example-budgeting-winter-coat}.
As the formulas do their work, the amount is then divided by the time period's length.
The \ac{bic} ``Months in 2019'' shows how many months are affected by the time period in the calendar year of the file, and in an analogical way does ``€ in 2019'' show the same.

%\begin{table}[hbtp]
%	\centering
%	\sffamily
%	\caption[Example for Budgeting a Winter Coat: Gray Columns]{Example for Budgeting a Winter Coat.
%		The values from the columns listed in \autoref{tab:example-budgeting-winter-coat} yield these entries in the gray columns.}
%	\label{tab:example-budgeting-winter-coat-graycolumns}
%	\begin{tabular}{|c|r|r|r|r|c|}
%		\hline
%		\footnotesize\textbf{Cat.-No.} &
%		\begin{minipage}[b][0.8cm]{1cm}\footnotesize\bfseries
%			Months\\
%			total
%		\end{minipage} &
%		\footnotesize\bfseries €/Month &
%		\begin{minipage}[b]{1.3cm}\footnotesize\bfseries
%			Months in\\
%			2019
%		\end{minipage} &
%		\begin{minipage}[b]{0.6cm}\footnotesize\bfseries
%			€ in\\
%			2019
%		\end{minipage}&
%		\footnotesize\bfseries Budgeted\\
%		\hline
%		\hline
%		\rowcolor{lightgray}
%		001 & 8 & 67.5 & 8 & 540.00 & No\\
%		\hline
%	\end{tabular}
%\end{table}

Now with \autoref{tab:example-budgeting-winter-coat} in mind, the process shown in \autoref{fig:example-budgeting-winter-coat-cashflows}, \autopageref{fig:example-budgeting-winter-coat-cashflows}, happens.
In theory, each month, 67.5\,€ get taken from the wallet and put into the virtual envelope each month.
This is continued until October 2019, when the entire amount of 540\,€ is budgeted.

\begin{figure}[hbtp]
	\centering
	\includegraphics{Gfx/Budgeting-Example-Winter-Coat-Cashflows.pdf}
	\caption[Cash Flows for the Winter Coat]{The cash flows for the winter coat.
	With each month, 67.5\,€ get put into the envelope.
	In theory, 
	In October 2019, its worth finally amounts to 540\,€ and the winter coat is fully budgeted.}
\label{fig:example-budgeting-winter-coat-cashflows}
\end{figure}

\subsubsection{Utility: Water}
\label{subsubsec:example-budgeting-utility-water}

A typical occurrence in a household's budget is the monthly expense for utilities such as water and electricity.
For budgeting the monthly expense for the water supply, let us take these facts \& figures into accounts:
\begin{itemize}
	\item The rate that is to be paid is 60\,€ per month.
	\item It is to be paid throughout the entire year.
\end{itemize}

As stated, this is a budget item with payments in each of the 12 months throughout 2019.
There is nothing to accumulate or to sum up.
Hence, a showcase like \autoref{fig:example-budgeting-winter-coat-cashflows} will not be presented here.
The relevant perspective is the evolution of the monthly payments and the sum prohibition, the latter being the crucial one.
For this, take a look at how the data is entered in \autoref{tab:example-budgeting-water-dataentry}.
For the sake of brevity, the table with the gray columns will be omitted.

Now to demonstrate the effect of sum prohibition, \autoref{tab:example-budgeting-utility-water-good} is provided.
This is the correct way to budget monthly expenses.
With monthly expense set to \codestuff{Yes} and sum prohibition set to \codestuff{No}, 60\,€ are budgeted and paid \emph{each} month.
At the end of time period, \ie December 2019, the amounts for all months in that time period do \emph{not} get summed up.

To demonstrate the effect of setting sum prohibition to the wrong value and therefor would could derail the budgeting of a expense with multiple payments, \autoref{tab:example-budgeting-utility-water-bad} is provided.
This is the false way to budget monthly expenses.
With monthly expense set to \codestuff{Yes} and sum prohibition set to \codestuff{No}, 60\,€ are budgeted and paid \emph{each} month.
At the end of time period, \ie December 2019, the amounts for all months in that time period do \emph{not} get summed up.

\begin{table}[hbtp]
	\centering
	\addtolength{\leftskip}{-1.0cm}
	\sffamily
	\caption[An Example for Budgeting a Winter Coat: Data Entry]{An Example for Budgeting a Winter Coat.
	The values from the bullet list at the beginning of  \autoref{subsubsec:example-budgeting-utility-water} are displayed in the corresponding columns.
	Note that the \ac{bic} monthly expense is set to \codestuff{Yes} and the \ac{bic} sum prohibition is set so \codestuff{No}.}
	\label{tab:example-budgeting-water-dataentry}
	\begin{tabular}{|c|l|l|l|l|c|c|c|l|}
		\hline
		\begin{minipage}{0.5cm}\footnotesize\bfseries
			No.
		\end{minipage} &
		\begin{minipage}{1.5cm}\footnotesize\bfseries
			Descr.
		\end{minipage} &
		\begin{minipage}[b]{1.0cm}	\footnotesize\bfseries	Amount
		\end{minipage} &
		\begin{minipage}[b][0.8cm]{0.8cm}\footnotesize\bfseries
			Start\\
			Date
		\end{minipage} &
		\begin{minipage}[b]{0.8cm}\footnotesize\bfseries
			End\\
			Date
		\end{minipage} &
		\begin{minipage}[b]{1.1cm}\footnotesize\bfseries
			Monthly\\
			Expense
		\end{minipage} &
		\begin{minipage}[b]{1.0cm}\footnotesize\bfseries
			Sum\\
			Prohib.
		\end{minipage} &
		\begin{minipage}[b]{0.8cm}\footnotesize\bfseries
			Done
		\end{minipage} &
		\begin{minipage}[b]{1.5cm}\footnotesize\bfseries
			Category
		\end{minipage}\\ 
		\hline
		\hline
		2 & Utility: Water & 720 & 2019-01 & 2019-12 & Yes & Yes & No & Utilities\\
		\hline
	\end{tabular}
\end{table}

\begin{table}[hbtp]
	\centering
%	\addtolength{\leftskip} {-2.25cm}
	\sffamily
	\caption[The Tables of Monthly Expenses for Water]{The table of monthly expenses for water.
	Each month, 60\,€ have to be paid to the utility provider.
	With sum prohibition set to \codestuff{Yes} (as it should be), the budgeted amounts will look like in the tables.}
	\label{tab:example-budgeting-utility-water-good}
	\begin{tabular}{|r|r|r|r|r|r|r|}
		\multicolumn{7}{l}{\footnotesize \emph{A look at the budgeted payments in January\ldots}}\\
		\hline
		\small 2019-01 & \small 2019-02 & \small 2019-03 & \ldots & \small 2019-10 & \small 2019-11 & \small 2019-12\\
		\hline
		Jan & Feb & March & \ldots & Oct & Nov & Dec\\
		\hline
		\hline
		60.00 & 0.00 & 0.00 & \ldots & 0.00 & 0.00 & 0.00\\
		\hline
		\multicolumn{7}{l}{~}\\
		\multicolumn{7}{l}{\footnotesize \emph{A look at the budgeted payments in March\ldots}}\\
		\hline
		60.00 & 60.00 & 60.00 & \ldots & 0.00 & 0.00 & 0.00\\
		\hline
		\multicolumn{7}{l}{~}\\
		\multicolumn{7}{l}{\footnotesize \emph{A look at the budgeted payments in December\ldots}}\\
		\hline
		60.00 & 60.00 & 60.00 & \ldots & 60.00 & 60.00 & 60.00\\
		\hline
	\end{tabular}
\end{table}

\begin{table}[hbtp]
	\centering
	%	\addtolength{\leftskip} {-2.25cm}
	\sffamily
	\caption[Budgeting Example: The Tables of Incorrectly Set Up Budgeted Monthly Expenses for the Item Utility: Water]{The table of monthly expenses for water.
	Each month, 60\,€ have to be paid to the utility provider.
	With sum prohibition incorrectly set to \codestuff{No}, the budgeted amounts will look like as follows and produce a false result.
	For the sake of brevity, only the month with the wrong sum is presented.}
	\label{tab:example-budgeting-utility-water-bad}
	\begin{tabular}{|r|r|r|r|r|r|r|}
		\multicolumn{7}{l}{\footnotesize \emph{A look at the budgeted payments in December\ldots}}\\
		\hline
		\small 2019-01 & \small 2019-02 & \small 2019-03 & \ldots & \small 2019-10 & \small 2019-11 & \small 2019-12\\
		\hline
		Jan & Feb & March & \ldots & Oct & Nov & Dec\\
		\hline
		\hline
		0.00 & 0.00 & 0.00 & \ldots & 0.00 & 0.00 & \textbf{720.00}\color{black!60!red}\ding{53}\\
		\hline
	\end{tabular}
\end{table}
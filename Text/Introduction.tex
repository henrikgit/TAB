\section{Introduction}
\label{sec:introduction}
% !TeX spellcheck = en_US

\subsection{First Things First - What Does This File Do?}
\label{subsec:first-things-first}

First and foremost, the spreadsheet file \sterm{Finances-2019.ods} is intended to help you to track and budget your finances, or rather, cash for 1 calendar year.
The file is free and can be viewed as a tool that you can use for managing your household's books and savings.\footnote{From here on forth, whenever \sterm{file} or \sterm{\tfn} is mentioned or referenced, it shall be interpreted synonymous with \sterm{Finances-2019.ods} or whichever year is written down in that file name.}
It is, in a sense, all about book-keeping or rather, \sterm{tracking} in sheets such as the ones displayed in \autoref{fig:introduction-tracking-sheet-screenshot}.
How much money is coming in, how much is going out and knowing where it is going.
As stated, its intended time-frame encompasses 1 calendar year.

\missingfigure{Tracking sheet screenshot}

\begin{comment}
\begin{figure}[htp]
	\centering
	\caption[Screenshot of the Tracking Sheet named \sheetname{Groceries}]{Screenshot of the Tracking Sheet \sheetname{Groceries}.
	d0f8934zsd.
	The values and therefor the diagrams are fake and engineered to make up some kind of scenario for the sake of the screenshot.}
	\label{fig:introduction-tracking-sheet-screenshot}
\end{figure}
\end{comment}

Should you choose to use the designated sheets for budgeting your expenses and income, you will be able to budget your cash, \eg for allocating a certain amount for spending on groceries each month, or on a single item (or 3) you intend to purchase in a few months or years.

\missingfigure{Budgeting sheet screenshot}

About data contained in the file:
\begin{specialnote}
	Neither does the file get automatically updated, nor does it receive or send data from/to any banking service, website or other kind of server.
	There is no connection ``programmed'' into it or so.
	The file relies on you to update it, which is essentially data entry.	
\end{specialnote}

\tfn was created in LibreOffice's spreadsheet program named \sterm{Calc}, hence I suggest you use it when working with this file.
However you may use whatever software you want.\footnote{About Excel: I do not own Excel, therefor I am not going to port it to Excel at the moment.
You can find more on that topic in \autoref{subsec:excel}, \autopageref{subsec:excel}.
If you use Excel with \tfn, please give me feedback of how it all works out!}

I have used this file and its countless previous versions for roughly 7 years by now.
I am not saying it is perfect and/or free of bugs, however I think it is ready to be put it online, 
I believe I have maintained, updated and improved it quite a lot in that time span, so with the best judgment possible, it is safe to say it can be used by others.
Nevertheless, it is provided to you as-is and you must bear all the responsibility when you use it.

\subsection{What Does The File Not Do?}
\label{subsec:not-included}

\tfn does \emph{not} do any of these things:
\begin{itemize}
	\item Analysis of different offers for portfolio calculations from different broker firms.
	\item Creating models of possible developments of pension plans with regards to tax optimization.
	\item Using and implementing some financial key performance indicators to model stock prices for buy-or-sell decision support.
\end{itemize}
I am not saying that any of these things are too specific or high-level and thus might be out of the question, but the file grew to be a tool for tracking and budgeting.
As my time is limited, I have no plans to integrate features for the items mentioned above in the immediate future.

Besides these things listed above, another aspect is the target group.
The term \sterm{household} was mentioned at the beginning of \autoref{subsec:first-things-first} and stands for whom this file is probably the most useful: a person or a group of persons, as opposed to (little) businesses.\footnote{But that should not deter you from using the file!}

\subsection{Basic Understandings}
\label{subsec:basic-understandings}

\subsubsection{Core Concepts}
\label{subsubsec:core-concepts}

The ``core concepts'', or rather, ``principles'' are as follows:
\begin{itemize}
	\item Data Entry:\\
	You will need to enter data for the use of this file.
	Or better put: for this file to have a purpose, you need to enter data into the \sterm{tracking sheets}, which are sheets that exist just to document you the amounts you spend and earn/receive.
	Please do so diligently and thoroughly.\footnote{And this is basically all there is to it.
	Using \tfn for managing your finances does not include rocket science.}
	\item Workload:\\
	My advice is to update the file at least every 2 weeks.
	It might take at least half an hour then.
	\item Human Error:\\
	You are human.
	It is inherently impossible to not make an error (unless you don't, which would be weird because you probably are not human).
	Do not get upset if you lost a receipt or forgot to write down a note with the exact price/amount of a purchase, activity or whatever.\footnote{This type of error is irreversible, unlike typos or logical errors within the file, \eg a wrong cell reference, incomplete sum and so forth.}
	The best course of action then is to enter a rough estimation.
	That is good enough and considerably better than to enter nothing.
	\item The Drift:\\
	An inevitable occurence when using the file is what I like to call \sterm{the drift}.
	It is a fancy term for the direct consequences of the human error (it most definitely is a human error that causes it, \ie an error within the file, and not a banking error).
	With time, you will undoubtedly notice a difference between the cash you have based on your file vs. the real-world amount.
	Accept it because otherwise you are fighting a fight you cannot win.
	To offset the drift, enter values in the \sterm{drift cells} in the sheet \sheetname{Tracker}.\footnote{For more on that, read through \autoref{subsec:different-values-finances.ods-vs-real-world}.}
\end{itemize}
With the best intention at heart, I can only ask you to follow and, more importantly, accept these principles.

\subsubsection{Implications}
\label{subsubsec:implications}

With \autoref{subsubsec:core-concepts} in mind, let us talk about what it actually means to work with \tfn.
%You should not use the file if one or multiple of the following points rings true.

\begin{itemize}
	\item You need to work with LibreOffice Calc.\footnote{Unless you use other software, that is.
	As mentioned before, more on that topic in \autoref{subsec:excel}, \autopageref{subsec:excel}.}
	\item You need to enter data into spreadsheet files.
	\item For that you need to collect receipts.
	In case you did not receive a receipt, you might need to write down (whether digitally or on paper) what you spent.
	This might be mitigated with the next point.
	\item You need to remember what you spent in case you neither received a receipt, nor made a note about the amount.
	\item You do not expect the file to somehow conjure up graphs and diagrams with insights on its own.
	Although the file does in fact possess a slim amount of ``intelligence'' through its formulas, it is all based on data \emph{you} entered.
	\item In case you would like to edit some formulas in the file, you would obviously have to learn editing spreadsheet files with Calc and understand how exactly the calculations work.
	\item Given that \tfn actually is a \codestuff{.ods}-file, this means that one does work on/in it while sitting in front of a computer/laptop.
	So this would not happen while being out and about.
	\item You are ok with facing numbers about your spending head-on.
	\item You have the discipline that is required to manage \tfn.
	\item In case you intend to use \tfn for managing finances of more than one person, I suppose all persons involved have to understand the points outlined above.
\end{itemize}

All the aforementioned bullet points entail important aspects which describe the use of the file.
If multiple points in the list represent something negative for you, I strongly suggest that you might find it more pleasant to use another program or app for managing your cash/finances overall.\footnote{You could then perhaps take a look at the tools page of the Personal Finance-subreddit:\\
\href{https://www.reddit.com/r/personalfinance/wiki/tools}{https://www.reddit.com/r/personalfinance/wiki/tools}.}
I do not intend to compete at all with any other cash tracking \& budgeting tools, let alone commercial software.
Why?
That is a good question and I will try to answer it in the following subsection.

\subsection{A Little Bit of History and Motivation}
\label{subsec:motivation-history}

Originally, I only used \tfn to track my cash spending, \ie what I spent on groceries, how much money I spent on eating out, clothing and so forth.
Over time, I started to add sections for tracking more and other kinds of categories.
I now track: my cash spending\footnote{I should note that I lump my cash spending and all other kinds of payments, \eg electronic, together.}, all other liabilities that represent cash outflows, my savings, my interests from these savings and other kinds of assets.
It grew into something that is possibly invaluable in my day-to-day life nowadays.
Yet perhaps there are still some errors I have not found so far or features that are missing or I have not thought of, that much would only be fair to add.

I created the file because I checked the landscape of budgeting software several years ago and found that I do not want to pay for managing my own finances.
And everything that was available did not quite interest me enough to fully use it and/or pay for it..
I was sure that with enough time and an eye for detail, I could build a tool for tracking my own finances.
And with even more time, coffee and bite to continues testing things and work out details, one could build some sort of formula/algorithm to expand the file's capabilities.
Well, you will find that there are lot of like-minded people sharing their knowledge and--for the lack of a better term--product in places such as the \href{https://reddit.com/r/personalfinance}{personal finance}-subreddit and similar platforms.

The second reason was some kind of concern for privacy-related thoughts.
It is also important to me that I understand what thing I am putting my personal finances into.
And on top of that I always want to know that I can limit the extent of risk and exposure.

So why do I put it online? 
Well, I think \tfn benefits from putting it online and reaching more people.
Ultimately, I share the file because I hope for a win-win outcome.
First of, this file might help others.
Secondly, I would like to receive feedback and suggestions.
If you are interested in that, you can jump to \autoref{sec:feedback} on \autopageref{sec:feedback}.

\subsection{Something Technical about \tfn}
\label{subsec:introduction-something-technical}

\subsubsection{Colors}
\label{subsubsec:colors}

There are several gray cells in the file.
You must not touch gray cells in the file.
Their interior color stands for \emph{do not edit}!

\subsubsection{Links/Hyperlinks}
\label{subsubsec:links-hyperlinks}

As stated on \autopageref{foreword}, the colored text indicates links to either URLs or targets within this document and can be clicked.
%They do \emph{not} get printed out and are just an overlay in your PDF viewer.
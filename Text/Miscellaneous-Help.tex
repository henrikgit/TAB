\section{Miscellaneous Help}
\label{sec:miscellaneous-help}
% !TeX spellcheck = en_US

\subsection{Budgeted Items Which Are Due}
\label{subsec:budgeted-items-which-are-due}

The sheet \sheetname{BudgetShoppingList} serves as a shopping list.
You will find the following items listed:
\begin{itemize}
	\item the items which are budgeted and can be bought in the current month,
	\item the items which are to be fully budgeted in the next month and in the month after that and
	\item all the budgeted items which you bought in the past months.
\end{itemize}

Note: the formulas in this sheet utilize the \ac{bic} \sterm{Done} that shows if a budget item has been bought or its monthly expense has finished (see \autoref{subsubsec:bic-done}).

\subsection{Income and Spending over a Certain Time Period}
\label{subsec:income-and-spending-certain-time-period}

The sheet \sheetname{TimePeriod} is for showing you the sums of daily cash outflows and inflows for the given time period.

For that, you need to type out all the names of your tracking sheets in the table which is to be found in that sheet.
After that you can enter the start and end date for an arbitrary time period and it will show produce the values.

As it has been mentioned a few times already, the underlying requirement is that you did do well on your tracking duties.

\subsection{Tracking Special Budget Items or Special Expenses}
\label{subsec:tracking-special-expenses}

Circling back to the surplus amounts, I would like to introduce another utilization of them: the sheet for tracking of expenses which are special, relatively speaking.\todo{ausfüllen}
If you want to, you can track your payments or saved amounts for certain goals or expenses in this sheet, \ie saving up 3000\,€.

Based on the data contained in this sheet, one could perhaps create or add a dashboard-like visualization in the sheet \sheetname{Overview}.

%Falls etwas überschuss da ist, kann man diesen theoretisch im nächsten monat extra verplanen! natürlich. nur wie möchte man den überblick halten, wenn man monatlich nach und nach stärker budgetiert, da man mehr mittel zu verfügung hat und dementsprechend verplanen kann?


\subsection{Navigation by Hyperlinking Sheets}
\label{subsec:navigation-hyperlinking-sheets}

One semi-related piece of advice is to use hyperlinks.
I originally thought to include this bit earlier in the document but I am sure the inherent characteristic is so familiar, this section is here for the sake of completeness.

Hyperlinks work just as you would expect from your browsing on the internet, so you should be familiar with their behavior.
You can insert one by using the shortcut \keystroke{Ctrl}+\keystroke{K}.
\begin{itemize}
	\item If you had already selected and copied some text, \loc should automatically use that text as the descriptor text and give you the opportunity to paste a link into the target field.
	\item Or if you had nothing in cache, you will have to enter both URL and the descriptor text.
\end{itemize}

If you intend to click on a link in \loc, \emph{first} hold \keystroke{Ctrl} and then click on it with your mouse.
You can change this setting in the options, by clicking on ``Tools'' in the task bar and then: Options \structurenext LibreOffice \structurenext Security.
Then, click on ``Buttons'' and remove the check mark next to ``Ctrl-click required to follow hyperlinks''.

There are a few more things to note:
\begin{itemize}
	\item There can be more than 1 hyperlink in a cell.
	\item Hyperlinks can target a site on the internet, a file on your machine or a specific cell in this or another sheet in this or another \loc-file.
\end{itemize}

\subsection{Forecasting}
\label{subsec:forecasting}

There is one other feature that I had originally planned to integrate: \emph{forecasting}.
Forecasting is the process of making predictions about future developments based on past and present data and most commonly by analysis of trends.\footnote{Cp. \href{https://en.wikipedia.org/wiki/Forecasting}{https://en.wikipedia.org/wiki/Forecasting}.}

So far I have not used trends and honestly, I think I was able to deduce the good-enough optimizations of budgeted amounts by myself.
And as stated multiple times throughout this guide, the file spans over 12 months (only), which obviously makes it impossible to form average values which account for values spanning over a longer time period than 12 months.

I am not opposed to integrate more sheets and calculations into \tfn, so if someone were to create some sheets for forecasting, they could perhaps be integrated in the future.

\subsection{Analysis of Your Data and Preparing for the Next Year}
\label{subsec:prepare-the-next-year-analysis}

Before you can move on to preparing \tfn for the next year, you should finalize the current/old file:
\begin{enumerate}
	\item Come December 31, the first ToDo-point on your list should be to save and close your current file.
	I suggest to not delete any data yet.
	\item Make a copy and rename the copy accordingly to the new year, \eg to \emph{Finances-2021.ods}.
	\item Open your (old/current) file and write down the last pieces of information in regards to the last day of the year: literally write down \codestuff{2019-12-31} into any cell that prints today's date, \eg in \sheetname{Budget}.
	Also look for cells that print the current month or day and fill in \codestuff{2019-12} or \codestuff{31}.
	\item Proceed in an analogical fashion for every cell that prints dates through formulas.
	Otherwise the next time you open that file will be later than December 31 of that ``old'' year and the formulas which rely rather, they do not show values which make sense., the cell that prints the date of that day which will probably somewhere in January.
\end{enumerate}

\begin{itemize}
	\item Take a really close look at the \sheetname{Budget} sheet and analyze the outcomes of the deviations throughout the whole year.
	You should put a focus on the 
	\begin{itemize}
		\item Kategorien mit den großen Monatsstandard-Anteilen beackern\todo{weiter}
		\item basierend auf den Inhalten die neuen Monatsstandards gestalten
		\item hinweis: könnte quasi unmöglich sein, da sehr viele einzelne budget-items übereinandergelegt wurden
		\item bei der Analyse sollte auch berücksichtigt werden, dass das Budget einerseits mit und ohne Abweichung gleichzeitig ausgewiesen wird - das heißt, es werden auch vorgehaltene, noch weiterhin zu sparende Beträge ausgewiesen.
		\item alles aus einem Pot
	\end{itemize}	
	\item Finally, save the file.	
\end{itemize}
\section{Budgeting}
\label{section:Budgeting}

Here is a short primer on budgeting:
\begin{quote}\small
	The process or definition of budgeting is the act of creating a plan to spend your money.
	Creating this spending plan allows you to determine in advance whether you will have enough money to do the things you need to do or would like to do. .\footnote{Cp. \url{https://www.mymoneycoach.ca/budgeting/}.}
\end{quote}

Why should you budget? A budget will help you plan for short-term expenses such as your monthly bills; mid-term expenses such as vacations; and long-term expenses such as buying a house, paying for a child’s college education or putting away money for retirement.
When you have an app, spreadsheet or notebook in front of you showing how much money you expect to make over the one month, six months, one year or five years — and how much of that money will be flowing out and how much you will have left to save each month — you’ll always know when you need to cut back on spending, when you can afford to loosen the reins and how long it will take to save for major goals or pay off debts.
If you’re not happy with the numbers, knowing what they are will help you take steps to improve your situation.
These steps might include paying off credit cards to increase your monthly cash flow; reducing your food expenses; or getting a promotion, switching companies, starting a side hustle or founding your own full-time business to make more money.\footnote{Cp. \url{https://www.investopedia.com/university/budgeting/basics1.asp}.}

As mentioned in \autoref{subsec:first-things-first} , if you like to budget your income and expenses, you need to use the budgeting sheets for this.
It is possible to create lots of different budget items which essentially represent either incoming or outgoing cash flows.
In other words, you can budget practically any kind of cash flow, regardless if you expect it to receive or spend any cash amount in the future.

\subsection{Budgeting for Expenses and Incoms}
\label{subsec:budgeting-expenses}

First and foremost, please do not touch the columns to the right of column \begriff{I}.
Considering their are more cells in the same block to the right of column \begriff{I}, I formatted their interior color gray which stands for \emph{DO NOT EDIT}!
And same goes for all the other columns continuing to the right of the gray area.

This subsection is for budgeting for both expenses and income, but for the sake of ``brevity'' we are going to focus on the expense budgeting.
Once you understood how to budget for an expense, you have the knowledge to budget for an income item as well. There is nothing new to explain here.
You would simply have go to the sheet titled \begriff{BudgetINCitems} and do it in an analogical way.

\subsubsection{Budgeting Basics}
\label{subsubsec:budgeting-basics}

You can create a budget for expenses by going to the sheet \begriff{BudgetEXPItems}.

Here are the steps you need to take to budget for an expense item.
Note: their order does not matter, just that you do \emph{all} of them!\
todo{evtl alle Schritte in eigene Sektionen?}
Except the first one, actually.
\begin{enumerate}
	\item This is the optional step: enumerate the item you are about to add.
	I suggest that you do because it is necessary for rearranging the items and who does not like a well-sorted list?
	\item Fill in its description or just name it.
	\item Enter the amount you plan to allocate to the item.\todo{Zeug wegen Gesamtbeträgen über mehrere Monate hinweg, \dh nicht nur Betrag für einen Monat - WERT für Beispiel angegeben}
	\item Enter the start date (\ie month) you want to begin budgeting for that item.
	The date you enter must have the format \programmcode{YYYY-MM}, \eg \programmcode{2019-02} which stands for February 2019.
	\item Enter end date (\ie month) in which you plan to purchase the item or the monthly expense stops.
	The date you enter must have the format \programmcode{YYYY-MM}, \eg \programmcode{2019-09} which stands for September 2019.
	\item About the column \begriff{monthly expense}:
	\begin{itemize}
		\item If the item is a single item (\eg a pair of shoes), you need to set the value to \programmcode{No}.
		\item If the item is an expense that you actually spend in \emph{each} month (\eg utilities, such as water, electricity etc.) on, you need to set the value to \programmcode{Yes}.
	\end{itemize}
	\item The column titled \begriff{sum prohibition} represents a switch for, like the name says, prohibiting summing up the budgeted amount or not.
	\begin{itemize}
		\item Put \programmcode{No} if the values are to ``stay'' in the months, \ie every month from February to September in 2019.
		budgeted amount is to be summed up in the end month or not.
		It says whether the amount in column C will be divided over the given time period or not.
		\item asdf\todo{explain what YES stands for}
		\item \emph{Warning:} You must put \programmcode{Yes} here if you have a monthly expense, otherwise all the amounts per month get summed up to their total in the last month, which would lead to strong deviations in each of the relevant months.
	\end{itemize}
	\item The column titled \begriff{Budgeted}
	\item Column \begriff{Category}  poijsad
\end{enumerate}
%TODO{Hier weitermachen}
\todo{hier weitermachen}

\subsubsection{Examples for Budgeting Expenses}
\label{subsubsec:examples-budgeting-expenses}

\subsection{Thoughts on Budgeting}
\label{subsubsec:thoughts-on-budgeting}

\subsection{Advice for Using Budgeting Sheets}
\label{subsec:advice-budgeting-sheets}

\subsection{Different Dates in Real Life}
\label{subsec:Different-Dates}

\begin{itemize}
	\item If you bought an item \emph{earlier} than you planned, backdate said item! Otherwise the budget will penalize you in the month it was dated for purchase and you will have made a huge minus in the month you actually purchases it.
	\item If you forgot to purchase 
\end{itemize}

\subsection{Total Deviation of Budget Categories}
\label{subsec:Total-Deviation}

\begin{itemize}
	\item In general, the budget categories get summed and lumped together each month.
	\item One could argue that the budget categories should be 
	\item Dealing with (delayed) deviations can result in a double-edged sword:
	\begin{itemize}
		\item On one hand, it feels kind of ``good'' to have a surplus for a budget item.\\
		\ldots in case this happens often, this item is clearly budgeted
		\item On the other hand, 
		\item \ldots  wrong.
	\end{itemize}
\end{itemize}


% !TeX spellcheck = en_US
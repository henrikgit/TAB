\section{Budgeting}
\label{section:Budgeting}

A short primer on budgeting:
\begin{quote}\small
	The process or definition of budgeting is the act of creating a plan to spend your money.
	Creating this spending plan allows you to determine in advance whether you will have enough money to do the things you need to do or would like to do. .\footnote{Cp. \url{https://www.mymoneycoach.ca/budgeting/}.}
\end{quote}

A budget will help you plan for short-term expenses such as your monthly bills; mid-term expenses such as vacations; and long-term expenses such as buying a house, paying for a child’s college education or putting away money for retirement.
When you have an app, spreadsheet or notebook in front of you showing how much money you expect to make over the one month, six months, one year or five years — and how much of that money will be flowing out and how much you will have left to save each month — you’ll always know when you need to cut back on spending, when you can afford to loosen the reins and how long it will take to save for major goals or pay off debts.
If you’re not happy with the numbers, knowing what they are will help you take steps to improve your situation.
These steps might include paying off credit cards to increase your monthly cash flow; reducing your food expenses; or getting a promotion, switching companies, starting a side hustle or founding your own full-time business to make more money.\footnote{From \url{https://www.investopedia.com/university/budgeting/basics1.asp}.}

As mentioned in \autoref{subsec:first-things-first}, if you like to budget your income and expenses, you need to use the budgeting sheets for this.
It is possible to create lots of different budget items which essentially represent either incoming or outgoing cash flows.
In other words, you can budget practically any kind of cash flow, regardless if you expect it to receive or spend any cash amount in the future.

\subsection{Budgeting for Expenses}
\label{subsec:budgeting-expenses}

Here are the steps you take to budget for an expense item.
Note: their order does not matter, just that you do \emph{all} of them.
\begin{enumerate}
	\item This is an optional step: enumerate the item you are about to add.
	\item Fill in its description or just name it.
	\item Enter the amount.
	\item Enter the month (it must have the format \programmcode{YYYY-MM}, \eg \programmcode{2019-02}) you want to start budgeting for it.
	\item Enter the month (it must have the format \programmcode{YYYY-MM}, \eg \programmcode{2019-09}) you intend to purchase the item.
	\item If the item 
\end{enumerate}
%TODO{Hier weitermachen}
\todo{hier weitermachen}

\subsection{Budgeting for Income}
\label{subsec:budgeting-income}

There is nothing new to explain here.
Simply go to the budgeting sheet for 

\subsection{Advice for Using Budgeting Sheets}
\label{subsec:advice-budgeting-sheets}

\subsection{Different Dates in Real Life}
\label{subsec:Different-Dates}

\begin{itemize}
	\item If you bought an item \emph{earlier} than you planned, backdate said item! Otherwise the budget will penalize you in the month it was dated for purchase and you will have made a huge minus in the month you actually purchases it.
	\item If you forgot to purchase 
\end{itemize}

\subsection{Total Deviation of Budget Categories}
\label{subsec:Total-Deviation}

\begin{itemize}
	\item In general, the budget categories get summed and lumped together each month.
	\item One could argue that the budget categories should be 
	\item Dealing with (delayed) deviations can result in a double-edged sword:
	\begin{itemize}
		\item On one hand, it feels kind of ``good'' to have a surplus for a budget item.\\
		\ldots in case this happens often, this item is clearly budgeted
		\item On the other hand, 
		\item \ldots  wrong.
	\end{itemize}
\end{itemize}


% !TeX spellcheck = en_US
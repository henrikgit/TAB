\section{Budgeting}
\label{section:Budgeting}
% !TeX spellcheck = en_US

Here is a short primer on budgeting:
\begin{quote}\small
	The process or definition of budgeting is the act of creating a plan to spend your money.
	Creating this spending plan allows you to determine in advance whether you will have enough money to do the things you need to do or would like to do. .\footnote{Cp. \url{https://www.mymoneycoach.ca/budgeting/}.}
\end{quote}

\subsection{Motivation for Creating a Budget}
\label{subsec:motivation-creating-budget}

Why should you budget? A budget will help you plan for short-term expenses such as your monthly bills; mid-term expenses such as vacations; and long-term expenses such as buying a house, paying for a child’s college education or putting away money for retirement.
When you have an app, spreadsheet or notebook in front of you showing how much money you expect to make over the one month, six months, one year or five years --- and how much of that money will be flowing out and how much you will have left to save each month --- you’ll always know when you need to cut back on spending, when you can afford to loosen the reins and how long it will take to save for major goals or pay off debts.
If you’re not happy with the numbers, knowing what they are will help you take steps to improve your situation.
These steps might include paying off credit cards to increase your monthly cash flow; reducing your food expenses; or getting a promotion, switching companies, starting a side hustle or founding your own full-time business to make more money.\footnote{Taken verbatim from \url{https://www.investopedia.com/university/budgeting/basics1.asp}.}

\subsection{Budgeting in Finances.ods}
\label{subsec:budgeting-in-finances.ods}

\subsubsection{Budgeting Basics}
\label{subsubsec:budgeting-basics}

As mentioned in \autoref{subsec:first-things-first}, if you like to budget your income and expenses, you need to use the budgeting sheets for this.
It is possible to create lots of different budget items which essentially represent either incoming or outgoing cash flows.
In other words, you can budget practically any kind of cash flow, regardless if you expect it to receive or spend any cash amount in the future.

As outlined in a few paragraphs above, the file utilizes the approach that the user plans his budget tailored to his needs.
To do so, one starts by writing down a list that consists of all the (possible) expense items.
So this is a process which implies that the user plans his expenses beforehand, not entirely but to the best of his abilities at that point in time.
And the budget as a whole gets more and more optimized to achieve the desired goals, work with the prioritized expense categories at the right time, account for relatively high one-time/emergency expenses and so forth.

\subsubsection{Budgeting for One Year}
\label{subsubsec:budgeting-for-one-year}

As mentioned in \autoref{subsec:first-things-first} and \autoref{subsec:tracking-column-date}, \tfn is created to be used for one calendar year.
There is nothing speaking against using the file for multiple (consecutive) years, except that the formulas that are built into it only check for the right month and day, thus rendering the calculated sums per month meaningless.

\subsubsection{The Envelope Method}
\label{subsubsec:budgeting-the-envelope-method}

The budgeting principle that is used in \tfn is the envelope method.
If you would like to know more about this way of budgeting, please visit your favorite search machine, but a short introduction can be found on \href{https://en.wikipedia.org/wiki/Envelope_system}{wikipedia}\footnote{URL: \href{https://en.wikipedia.org/wiki/Envelope_system}{https://en.wikipedia.org/wiki/Envelope{\_}system}.}.
Obviously we use it in its digital or virtual form, as there is no envelope being used within the file but the principle is adhered to nonetheless.

The key to translating the envelope method to a virtual perspective is that the envelope could be interpreted as an extra budgeting account you could put your money into.
You have two choices:
\begin{enumerate}
	\item The money may remain in whatever account you have included in the tracking.
	\item You could transfer some amount \( X \) for a month with budgeted items (which would be practically every month) to that virtual envelope, \ie some kind of special account for budgeting needs.
\end{enumerate}

To clarify the second choice and its implications:
\begin{itemize}
	\item One would need to have a spare account.
	\item One would need to perform the necessary calculations to get the right amount for the month.
	\item Ultimately, one would need to process all these arguably small money transfers for the amounts for budgeted items which would be due that month.\footnote{And transferring them in bulk would definitely lead to the system being to inflexible, since you would have to answer the questions of timing and the composition (which items would have to be included) for a transfer.
	All in all, it is probably best to let that topic rest.}
\end{itemize}

\subsection{Budgeting for Expenses}
\label{subsec:budgeting-expenses}

You can create a budget for expenses by going to the sheet \sheetname{BudgetEXPItems}.
You will observe there are gray cells to be found.
As mentioned in \autoref{subsec:introduction-something-technical}, they are not to be edited.
Furthermore:
\begin{center}\sffamily
	Do not edit the cells to the right of the gray cells.
\end{center}

For creating an entry for an expense you want to budget, you need to fill out columns and you should read through \autoref{subsubsec:budgeting-item-column-number} to \autoref{subsubsec:budgeting-item-column-category}.
It is important that you fill out \emph{all} columns!
Well, except the first one, actually.
These columns have the magical name \ac{bic} and--with some liberty--could be interpreted as steps you need to take to create a budget.

\subsubsection{BIC: Number}
\label{subsubsec:budgeting-item-column-number}

This the optional step: enumerate the item you are about to add.
I suggest that you do because as it may become necessary to rearrange the items.
Also, who does not like a well-sorted list?? :)

\subsubsection{BIC: Description}
\label{subsubsec:budgeting-item-column-description}

Just name the item and/or fill in a few details.

\subsubsection{BIC: Amount}
\label{subsubsec:budgeting-item-column-amount}

In short: enter the amount you plan to allocate to the item.
If you are about to enter data for an expense item with multiple payments for a time period longer than 1 month, you have 2 possibilities:
\begin{enumerate}
	\item You enter the total amount for the year and let \tfn do the work to have it spread out evenly from start to end.
	\item Or you enter multiple time periods with different kinds of lengths and amounts to ultimately have the right total for all the payments over the desired time period.
	There are so many ways you can choose to do this
\end{enumerate}

\subsubsection{BIC: Start Date}
\label{subsubsec:budgeting-item-column-start-date}

Enter the start date you want to begin budgeting for that item.
The date you enter must have the format \programmcode{YYYY-MM}, \eg \programmcode{2019-02} which stands for February 2019.

\subsubsection{BIC: End Date}
\label{subsubsec:budgeting-item-column-end-date}

Enter the end date in which you plan to purchase the item or the monthly expense stops.
The date you enter must have the format \programmcode{YYYY-MM}, \eg \programmcode{2019-09} which stands for September 2019.

\subsubsection{BIC: Monthly Expense}
\label{subsubsec:budgeting-item-column-monthly-expense}

The \ac{bic} titled \begriff{monthly expense} is for the following distinction:
\begin{itemize}
	\item If the expense item consist of a single purchase/item/money transfer, you need to set the value to \programmcode{No}
	An example would be the purchase of a pair of shoes.
	\item If the item is an expense that you actually spend money on in multiple months throughout the year, you need to set the value to \programmcode{Yes}.
	An example would be paying for \eg utilities, such as water, electricity etc., on multiple occasions throughout the year.
\end{itemize}

There is also a clarification to be made about reoccurring expenses which are at intervals that do not equal a month's length.
This will be elaborated on in \autoref{subsubsec:thoughts-non-monthly-expenses}, \autopageref{subsubsec:thoughts-non-monthly-expenses}.

\subsubsection{BIC: Sum Prohibition}
\label{subsubsec:budgeting-item-column-sum-prohibition}

This \ac{bic} is for prohibiting if the amount for an expense item is summed up or not.
So why is there a column which aims to prohibit summing
This column mostly exists because of the limitations that I ran into with using spreadsheet software and, frankly speaking, the way I conceptualized budgeting over multiple months.
This 



Note: this column/switch only has an effect if the time period is longer than 1 month.

First of, a little detour to provide the right theoretical background: as you will observe, if you budget an entry for a time period longer than 1 month, the amount gets divided evenly by the number of months, \ie the length of the time period.
So\ldots should you fill in \programmcode{Yes} or \programmcode{No}?
\begin{itemize}
	\item \programmcode{Yes} obviously activates the prohibition of summing up the budgeted amount.
	Put \programmcode{Yes} if the values are to ``stay'' in the months, \ie every month from March to October in 2019.
	\item \programmcode{No}
	So if the time period of the budgeted expense is longer than 1 month, the monthly values of the budgeted expense ``stay'' in their months.
\end{itemize}
Again, please do read \autoref{subsec:examples-budgeting-expenses} starting on \autopageref{subsec:examples-budgeting-expenses} for more clarification.



At the end, one would take all the money for that out of that envelope because if you add up all the monthly values, you end up with 800\,€.
At this point in time, the expense item is \begriff{budgeted} and can be purchased/billed.
But: what if you have 

spend it on the budgeted expense, which has now been fully budgeted.

The way \tfn works is that with each month, it stays true to the envelope method and 

The column titled \begriff{sum prohibition} represents a switch for, like the name says, prohibiting summing up the value in the \ac{bic} amount or not.


The switch says whether all these divided values get summed up in the last month, which ultimately makes up the original amount.



\emph{Lesson:} You must put \programmcode{Yes} here if you have a monthly expense, otherwise all the amounts per month get summed up to their total in the last month, which would lead to strong deviations in each of the relevant months.

Reoccurring items, such as a monthly expenses regardless if true or fake--need a 

		budgeted amount is to be summed up in the end month or not.

\subsubsection{BIC: Budgeted}
\label{subsubsec:budgeting-item-column-budgeted}

This column serves to enter if the budget item has been fully budgeted or not.
\begin{itemize}
	\item Put \programmcode{Yes} if the end date is reached or has been in the past (and you forgot to update this budget item).
	\item Put \programmcode{No} if the 
\end{itemize}

Once you 

\subsubsection{BIC: Category}
\label{subsubsec:budgeting-item-column-category}

Column \begriff{Category} poijsad

\subsection{Budgeting for Income}
\label{subsec:budgeting-income}

This subsection of the guide simply exists if you, the reader, decided to read up on budgeting for income.
Because once you understood how to budget for an expense, you have the knowledge to budget for an income item as well.
At this point, I do not think there is anything new to explain in regards to budgeting income items.
You would simply have go to the sheet \sheetname{BudgetINCitems} and do it in an analogical way.

\subsection{Thoughts on Budgeting}
\label{subsubsec:thoughts-on-budgeting}

\subsubsection{Thoughts on Non-Monthly Expenses}
\label{subsubsec:thoughts-non-monthly-expenses}

This part is in regards to \autoref{subsubsec:budgeting-item-column-monthly-expense}, where reoccurring expenses which are to be paid at non-monthly intervals (\eg quarterly) were mentioned.
There are further elaborations to be made on the possibilities of non-monthly expenses.
\begin{itemize}
	\item You can set their start and end date to span over the whole year (or 12 months): the expense item would occur after 12 months and it would be the total of all single payments.
	There would be a deviation in all months, but the yearly total deviation equals 0.
	The \begriff{sum prohibition} switch would have to be set to \programmcode{Yes}.
	\item For each payment, \ie the expense is regarded as a single purchase and needs the \begriff{sum prohibition} switch set to \programmcode{No}.
	Hence one would have to create multiple entries for the whole group of quarterly payments.
	\begin{itemize}
		\item One could set the start date for all payments to the month of today or whenever one would want to start budget for it.
		Note: this would lead to an increased budgeting load in the ensuing months right after the start date.
		\item One could set the start date to the month after a payment.
		Explanation: let us assume the first payment would be due in march, thus the time period would be \programmcode{2018-01} to \programmcode{2018-03}.
		Accordingly, the next budget item could start in \programmcode{2018-04}.
	\end{itemize}
\end{itemize}

\subsubsection{Thoughts on Different Dates in Real Life}
\label{subsubsec:thoughts-different-dates}

There are a few possibilities which could make this topic bear relevance:
\begin{enumerate}
	\item If you purchased an item \emph{earlier} than the (planned) end date, I suggest that you backdate its end date.
	Otherwise the budget will penalize you in the month it was dated for purchase.
	Seeing how this action, \ie purchasing item(s), is a not a monthly expense, the monthly deviation is the one that matters, not the yearly one.
	Therefor, if you do not backdate the you will have made an unnecessary deviation in the month you actually purchases it.
	Dev in the new month and the higher 
	\item If you end up not purchasing something in the right month, that is on you.
	But that is ok and you could be happy about the positive effect on the deviation.
	\item The debit is withdrawn too late or too early from your account and by being the thorough tracker you certainly strive to be, you treat the dates as they occur in real life and ultimately have a unnecessary deviation, but well, that is life.
	Regardless, you update your list to reflect your view 	(because who does not like a well-maintained list, right??).
\end{enumerate}

As mentioned \autoref{subsec:tracking-column-date}, there are 2 ways (at least the ones I listed) that dates can be used:
\begin{enumerate}
	\item A date for the day something was bought/ordered.
	Regardless of any kind of money transfer, this is the date the purchase was planned for.
	\item With regards to any kind of money transfer, there is also date--in case one uses some form of electronic payment (\eg a card)--on which the amount is actually transferred from your account.
\end{enumerate}

\subsubsection{Thoughts on Delayed Budgeting}
\label{subsubsec:thoughts-delayed-budgeting}

Say one the following situation occurs:
\begin{itemize}
	\item It is the last day of the month.
	\item One would go on a spontaneous shopping trip.
	\item You know you have some certain, yet unspecified, amount to spend.
	\item One would spend that amount.
\end{itemize}

The nice way of going on with this is what I would call \begriff{delayed budgeting}.
But this is not a 

\subsubsection{Thoughts on Deviations}
\label{subsubsec:thoughts-deviations}

\begin{itemize}
	\item In general, the budget categories get summed and lumped together each month.
	\item One could argue that the budget categories should be 
	\item Dealing with (delayed) deviations can result in a double-edged sword:
	\begin{itemize}
		\item On one hand, it feels kind of ``good'' to have a surplus for a budget item.\\
		\ldots in case this happens often, this item is clearly budgeted
		\item On the other hand, 
		\item \ldots wrong.
	\end{itemize}
\end{itemize}

\subsection{Examples for Budgeting Expenses}
\label{subsec:examples-budgeting-expenses}

\subsubsection{Winter Coat}
\label{subsubsec:example-budgeting-expense-winter-coat}

For the example of budgeting a winter coat, let us take these facts \& figures into accounts:
\begin{itemize}
	\item We want to budget for the purchase of a super awesome winter coat which could cost up to with a value of 800\,€.
	\item It is March and it possible to buy the coat in October.
	Then the stores definitely have the new merchandise on the shelves.
\end{itemize}
The first columns in the row for that budget item could look like shown in \autoref{tab:example-budgeting-expense-winter-coat-dataentry}.
These entries yield the values shown in the gray columns in \autoref{tab:example-budgeting-expense-winter-coat-graycolumns}.

\begin{table}[htbp]
	\centering
	\addtolength{\leftskip} {-1cm}
	\sffamily
	\caption[Example for Budgeting a Winter Coat: Data Entry]{Example for Budgeting a Winter Coat.
	The values from the bullet list above are displayed in the corresponding columns.}
	\label{tab:example-budgeting-expense-winter-coat-dataentry}
	\begin{tabular}{|l|l|l|l|l|l|l|l|l|}
		\hline
		\footnotesize\textbf{No.} & \rotatebox{0}{\footnotesize\textbf{Descr.}} & \rotatebox{0}{\footnotesize\textbf{Amount}} & \rotatebox{0}{\footnotesize\textbf{Start Date}} & \rotatebox{0}{\footnotesize\textbf{End Date}} & \rotatebox{0}{\footnotesize\textbf{Monthly Exp.}} & \rotatebox{0}{\footnotesize\textbf{Sum Prohib.}} & \rotatebox{0}{\footnotesize\textbf{Budgeted}} & \rotatebox{0}{\footnotesize\textbf{Category}}\\ 
		\hline
		\hline
		1 & Winter Coat & 800 & 2019-03 & 2019-10 & 100 & No & No & Clothing/Acc.\\
		\hline
	\end{tabular}
\end{table}

\begin{table}[htbp]
	\centering
	\sffamily
	\caption[Example for Budgeting a Winter Coat: Data Entry]{Example for Budgeting a Winter Coat.
	The values from the bullet list above are displayed in the corresponding columns.}
	\label{tab:example-budgeting-expense-winter-coat-graycolumns}
	\begin{tabular}{|l|l|l|l|l|l|}
		\hline
		\footnotesize\textbf{Cat.-No.} & \footnotesize\textbf{Months tot.} & \footnotesize\textbf{€/Month} & \footnotesize\textbf{Months in 2019} & \footnotesize\textbf{€ in 2019} & \footnotesize\textbf{Budgeted}\\
		\hline
		\hline
		\rowcolor{lightgray}
		001 & 8 & 100 & 8 & 800 & No\\
		\hline
	\end{tabular}
\end{table}

Now with \autoref{tab:example-budgeting-expense-winter-coat-graycolumns} in mind, let us circle back to \autoref{subsubsec:budgeting-item-column-sum-prohibition} and talk about its consequences for budgeting the winter coat.
As we have just read, we put 100\,€ into the virtual envelope\footnote{See \autoref{subsubsec:budgeting-the-envelope-method}.} per month.
This whole concept is displayed in \autoref{tab:bic-sum-prohibition-clarification}.

\begin{table}[htp]
	\centering
	\caption[Example for Sum Prohibition]{An Example for Sum Prohibition.
		The respective values }
	\label{tab:bic-sum-prohibition-clarification}
	\begin{tabular}{ccrr}
		\toprule
		\begin{minipage}[b]{1.0cm}
		Nr. of\\
		Month	
		\end{minipage}
		& \begin{minipage}[b]{1.4cm}
			Calendar\\
		 	Month	
		 \end{minipage}
		&
		\begin{minipage}[b]{2.25cm}
			Amount going\\
			to envelope	
		\end{minipage}
		 & 
		\begin{minipage}[b]{2.0cm}
			Total worth\\
			in envelope	
		\end{minipage}\\
%		\cmidrule(l{5.5pt}r{5pt}){2-2}
		\cmidrule(r){1-1}
		\cmidrule(lr){2-2}
		\cmidrule(lr){3-3}		
		\cmidrule(lr){4-4}
		1 & 2019-03 & 100 & 100\\
		2 & 2019-04 & 100 & 200\\
		3 & 2019-05 & 100 & 300\\
		4 & 2019-06 & 100 & 400\\
		5 & 2019-07 & 100 & 500\\
		6 & 2019-08 & 100 & 600\\
		7 & 2019-09 & 100 & 700\\
		8 & 2019-10 & 100 & 800\\
		\bottomrule
	\end{tabular}
\end{table}

\subsubsection{Utility: Water}
\label{subsubsec:example-budgeting-utility-water}


\subsection{Budgeting in Finances.ods}
\label{subsec:budgeting-in-finances.ods}
% !TeX spellcheck = en_US

The file uses the approach that you put all available funds in one pot and then budget them.

\subsubsection{Basics before You Start}
\label{subsubsec:budgeting-basics}

If you like to budget your income and expenses, you need to use the budgeting sheets for this.
It is possible to create lots of different budget items which essentially represent either incoming or outgoing cash flows.
In other words, you can budget practically any kind of cash flow, regardless if you expect it to receive or spend any cash amount in the future.

As outlined in a few paragraphs above, the file work with the approach that you plan your budget tailored to your needs.
\begin{enumerate}	
	\item To do so, you could start by writing down a list that consists of all the expenses that occur every month or every few months.
	These are (fixed) monthly expenses.
	\item Also write down typical expenses, such as eating out, going to sport events and other things you do or spend money on regularly.
	This list will serve for what I like to call \emph{placeholders}.
	\item Then think one-time expenses like a vacation, or a (new) car you would like to buy, or a new winter coat.
	\item Finally, you decide on the timing, \emph{when} you plan to buy these one time-expense items.
\end{enumerate}

So this is a process which implies that you plan your expenses beforehand.
Not entirely or perfectly, but to the best of your abilities at that point in time, while looking forward.
And if big changes or complications happen or various uncertainties arise, you definitely ought to rework these plans.

\subsubsection{Budgeting for One Year}
\label{subsubsec:budgeting-for-one-year}

As mentioned in \autoref{subsec:first-things-first} and \autoref{subsec:tracking-column-date}, \tfn is created to be used for one calendar year.
There is nothing speaking against using the file for multiple (consecutive) years, except that the formulas that are built into the tracking sheets only check for the right month and day, thus rendering the calculated sums per month meaningless.

\subsubsection{The Envelope Method}
\label{subsubsec:budgeting-the-envelope-method}

The budgeting principle that is used in \tfn is the envelope method.
If you would like to know more about this way of budgeting, please visit your favorite search machine, but a short introduction can be found on \href{https://en.wikipedia.org/wiki/Envelope_system}{wikipedia}\footnote{URL: \href{https://en.wikipedia.org/wiki/Envelope_system}{https://en.wikipedia.org/wiki/Envelope{\_}system}.}.
The system is of relevance if you are either unable or unwilling to buy an item in this very moment, \ie you plan to buy it in a few weeks or months.
In theory, the envelope method works like this:
\begin{enumerate}
	\item You write the name and total cost of an expense onto an envelope.
	\item Ideally, you put the same amount into the envelope each month.
	If you are not able to, you put any amount you wish anytime into the envelope.
	\item When the worth in the envelope reaches the values written on it, you can take it out.
\end{enumerate}

The principle is obviously used in its digital or virtual form, as there is no envelope being used within the file but the principle is adhered to nonetheless.
Secondly, it was ``enhanced'' as dates are being used.

The key to translating the envelope method to a virtual perspective is that the envelope could be interpreted as an extra budgeting account you could put your money into.
You have two choices:
\begin{enumerate}
	\item The money may remain in whatever account you have included in the tracking.
	\item You could transfer some amount \( X \) for a month with budgeted items (which would be practically every month) to that virtual envelope, \ie some kind of special account for budgeting needs.
\end{enumerate}

I strongly suggest to use the first way of doing this, as the second one is too complicated due to its implications.\footnote{1. You would need to have a spare account.
2. You would need to perform the necessary calculations to get the right amount for the month.
3. You would need to process all these arguably small money transfers for the amounts for budgeted items which are due that month.
4. Or transferring them in bulk would lead to even more annoying questions, because how exactly would one go about creating such a bulk transfer?}

\subsubsection{Deviation}
\label{subsubsec:deviation}

First of, the concept of \sterm{deviation} in the context of budgeting and \tfn is to be understood as the difference between the planned amount of an expense and its real-world amount.
The file automatically computes the surplus or deficit each month.

Secondly, the formulas in the budgeting sheets work this way:
\begin{enumerate}
	\item They look for the budgeted total in a category in month \( M \).
	Let us imagine we are dealing with the category clothing \& accessoires:
	\begin{itemize}
		\item The budgeted amount for a new pair of shoes is 70\,€.
		\item The budgeted amount for a new jacket is 100\,€.
	\end{itemize}
	This makes a budgeted total of 170\,€ for that month \( M \).
	\item Once month \( M \) is over, you are in month \( M+1 \).
	Because you do the tracking correctly, \tfn also is aware of what you actually spent on that new pair of shoes and your new jacket in \( M \).
	The amount for both items is totaled.
	\item This value is then compared to the budgeted total from before, \ie 170\,€.
\end{enumerate}

A look at the possible outcomes of that comparison:
\begin{itemize}
	\item If the budgeted total and the spent total equal each other, the deviation obviously equals 0 as well.
	\item If the amount spent is larger, \eg 190\,€, than the budgeted 170\,€, the deficit amount to \( 190-170 = 20\)\,€ and will be subtracted from the budget of the following month \( M+1 \).
	Perhaps view this as a \sterm{penalty}.
	\item Vice versa, if you spent 150\,€, you made a surplus of \( 170-150 = 20 \)\,€, which goes into the budget of month \( M+1 \).
	This is a \sterm{reward}. :)
\end{itemize}

\subsubsection{Using Deviation for Planning Your Budget}
\label{subsubsec:using-deviation-while-planning-your-budget}

The logic outlined in \autoref{subsubsec:deviation} is used for every month.
It is crucial for creating the budget for future months.
When you have finished writing down the details for your expense item, you will notice that the header up top will change.
It automatically shows you the \emph{planned} difference between the expenses and income of the months to come.
\begin{specialnote}
	While you work on your budget items, you will notice that the cells up top (``header'') change.
	This can be extremely helpful if you use it, as it is real-time feedback in regards to your budgeting.
	If you budget too strongly against your available funds, you essentially make a plan to overspend, hence you will accrue debt and the header will reflect that.
	In this case, you should change your plans accordingly.
\end{specialnote}

Now that you read a good bit about budgeting, you might finally be wondering why you should make the effort to create a budget at all as the deviation basically takes care of it already.
And on top of that, why should you think about it, if you just want to be spontaneous and buy something, how does that factor in?
Well, although I intentionally avoided to about personal angles, all of this obviously is about personal finance, so I will have a go at it from a semi-personal angle.
But I wrote about this already, so please go to \autoref{subsec:motivation-creating-budget} on \autopageref{subsec:motivation-creating-budget}. :)
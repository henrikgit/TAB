\section{Editing Finances.ods}
\label{sec:editing-finances.ods}
% !TeX spellcheck = en_US

While I would like to say that the file is pretty solid and works alright, its value is clearly dependent on the content it serves.
If you find that some tracking categories are missing or its structure overall should be changed, it is on you to adjust some sheets.
\begin{itemize}
	\item For adding or removing tracking categories to your liking, consult \autoref{subsec:editing-tracking-categories}.
	\item For changing the order/structure of said categories, go to \autoref{subsec:editing-order-of-tracking-categories}.
\end{itemize}

\subsection{Editing Sheets in General}
\label{subsec:editing-general-sheet-editing}

You can simply go ahead and do one of the following things:
\begin{itemize}
	\item Rename a sheet.
	\loc will ``see'' that new name and use it automatically in a formula, if you have already referenced a cell in that sheet in a formula.
	This means you will not have to update any existing formulas which were using the just-changed sheet name.
	\item Change the order of the sheets.
\end{itemize}

\subsection{Editing Content in Regards to Tracking}
\label{subsec:editing-content-in-regards-to-tracking}

There are several aspects relevant to editing content which has an influence on the tracking mechanism in \tfn:
\begin{itemize}
	\item A sheet which contains expenses or income can be edited.
	\item A tracking category and/or class can be changed.
	\item The order of tracking categories and/or classes is possible to edit.
\end{itemize}

\subsubsection{Editing Tracking Sheets}
\label{subsubsec:editing-tracking-sheets}

I structured their layout so the relevant cells for data entry are relatively fast to reach.
But if you like to edit the tracking sheets to change their (incredibly complex) design, you can do so.
Re-design them as you wish and please make sure to read \autoref{subsubsec:checking-correcting-formula-references}.

\subsubsection{Editing Tracking Categories}
\label{subsubsec:editing-tracking-categories}

When you want to edit anything in regards to the categories, you must start with the tracking categories.

\subsubsection{Renaming a Tracking Category}
\label{subsubsec:renaming-a-tracking-category}

If you want to rename a tracking category, you must work through the following list:
\begin{itemize}
	\item If you rename the tracking category, you must check all entries you made in the the \ac{bic} \codestuff{Category}.
	Obviously the whole tracking and budgeting aspect only makes sense if these category names in both sheets match.
	For more information, please go back to \autoref{subsubsec:bic-category}, \autopageref{subsubsec:bic-category}.
	\item Other than that, there is no need to change Calc's formulas, as described in \autoref{subsec:editing-general-sheet-editing}.
\end{itemize}

\subsubsection{Adding a Tracking Category Or Tracking Section}
\label{subsubsec:adding-a-tracking-category-section}

With \tfn being open and accessible, you can easily add a category or whole section for tracking purposes.
For this, work through the following list:
\begin{itemize}
	\item Go to the relevant row in the sheet \sheetname{Tracker}.
	\item In case you are not aware, there are 2 ways to insert a row:
	\begin{enumerate}
		\item Left-click on any content and then use the keyboard shortcut \keystroke{CTRL}+\keystroke{+} (the plus sign).
		Then click on the 3rd entry from the top, which says \codestuff{Entire Row}.
		The row gets then inserted \sterm{above} the highlighted row.
		\item Right-click on the row number (by that I mean the actual number from \loc's GUI which is highlighted) and then on \codestuff{Insert Rows Above} or \codestuff{Insert Row Below}.
	\end{enumerate}
	\item You should enter the new name for the category you just created.
	\item You also have to create a new sheet if you added a new tracking category.
	For naming that sheet, please consider the advice from \autoref{subsec:tracking-categories}, \autopageref{subsec:tracking-categories}.
	\item Finally, read through \autoref{subsubsec:checking-correcting-formula-references}, \autopageref{subsubsec:checking-correcting-formula-references}.
\end{itemize}

\subsubsection{Removing a Tracking Catergory or Tracking Section}
\label{subsubsec:removing-a-tracking-category-section}

For removing a tracking category or a whole section, I believe there is no extra section necessary and I hope it is sufficient to point you \autoref{subsubsec:adding-a-tracking-category-section}.
Simply apply the steps from there in analogical fashion.

\subsubsection{Editing The Order of Tracking Categories}
\label{subsec:editing-order-of-tracking-categories}

If you would like to have a different order of the tracking categories in the sheets \sheetname{Tracker} and \sheetname{Budget}, you can of course do so.
Editing the order of tracking categories builds upon \autoref{subsec:editing-tracking-categories} and is basically the same in terms of filling out content in the sheet \sheetname{Tracker}.
If you changed 

\subsubsection{Checking and Correcting Formula References After Changing the Tracker Sheet}
\label{subsubsec:checking-correcting-formula-references}

Once you changed content in the sheet \sheetname{Tracker}, you must check and possibly correct the formula references to the sheets (which rely on the system of the tracking categories).
These include:
\begin{itemize}
	\item \sheetname{Budget}
	\item \sheetname{BudgetEXPItems}
	\item \sheetname{BudgetINCItems}
	\item \sheetname{Budget{\_}real}
	\item \sheetname{Budget{\_}upco}
	\item \sheetname{Budget{\_}EXPCat}
	\item \sheetname{Budget{\_}INCCat}
\end{itemize}

\subsection{Editing Budgeting Content}
\label{subsec:editing-budgeting-content}

\subsubsection{Editing Budgeting Sheets}
\label{subsubsec:editing-budgeting-sheets}

I strongly suggest you do not edit any budgeting sheets.\todo{weiter}

\subsubsection{Editing Budgeting Categories}
\label{subsubsec:editing-categories-for-budgeting}

As stated in \autoref{subsubsec:bic-category}, \autopageref{subsubsec:bic-category}, writing something in the \ac{bic} ``Category'' is rather selecting an entry in a given list than literally writing something.
For this list to change in terms of budgeting categories, there are no extra steps.
As stated in \autoref{subsec:tracking-categories}, the budgeting categories are simply mirroring the tracking categories, hence for all things ``editing budgeting categories'' you must work through \autoref{subsec:editing-tracking-categories}.
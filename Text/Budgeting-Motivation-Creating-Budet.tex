\subsection{Motivation for Creating a Budget}
\label{subsec:motivation-creating-budget}
% !TeX spellcheck = en_US

Why should you budget? A budget will help you plan for short-term expenses such as your monthly bills; mid-term expenses such as vacations; and long-term expenses such as buying a house, paying for a child’s college education or putting away money for retirement.
When you have an app, spreadsheet or notebook in front of you showing how much money you expect to make over the one month, six months, one year or five years --- and how much of that money will be flowing out and how much you will have left to save each month --- you’ll always know when you need to cut back on spending, when you can afford to loosen the reins and how long it will take to save for major goals or pay off debts.
If you’re not happy with the numbers, knowing what they are will help you take steps to improve your situation.
These steps might include paying off credit cards to increase your monthly cash flow; reducing your food expenses; or getting a promotion, switching companies, starting a side hustle or founding your own full-time business to make more money.\footnote{Taken verbatim from \url{https://www.investopedia.com/university/budgeting/basics1.asp}.}

So all in all, if you want to, you can plan your future spending so that you know when your planned expenses might exceed your planned income.
If you want to, you can plan for future expenses such as a vacation, on top of the rather usual expenses such as buying groceries, clothes and the monthly payment on your car lease.

\begin{specialnote}
	If you are not interested in budgeting by now, please skip this section and continue with \autoref{sec:editing-finances.ods}, \autopageref{sec:editing-finances.ods}.
\end{specialnote}


\section{Tracking}
\label{sec:Tracking}
% !TeX spellcheck = en_US

Clearly the most riveting part of the guide.
It is probably the most mundane part of book-keeping, but definitely even more important.
This insanely complex topic can be summed up with: \emph{do the damn tracking}.
This it is.
You are done with the broadest understanding about tracking and can go on with it in \tfn. :)
But perhaps you feel like continue reading\ldots

\subsection{How To Enter Tracking Data}
\label{subsec:enter-tracking-data}

There are \begriff{tracking sheets} for entering data on your expenses and income.
These sheets simply contain data which is to be put into 3 columns: date, description and amount.

\subsubsection{Tracking Column: Date}
\label{subsec:tracking-column-date}

Date: Enter the day the expense occurred.
You can do this in two ways:
\begin{enumerate}
	\item Put the date for the day you bought it or ordered it online.
	\item Or--in case you use electronic cash (a card)--the date its value is actually transferred from your account.
	This might happen in the case of weekends, bank holidays or other kinds of service delays for money transfers.
\end{enumerate}
I strongly suggest you use the first kind of date, as the date represents the better point in time, \ie the one you planned for.\footnote{This topic will be referenced and elaborated on later.}
Regardless, you should stay consistent in the method you choose.
And:

\begin{center}
	Do not lie about the date.
\end{center}

You may use any format, but I suggest using \programmcode{YYYY-MM-DD} based on the international standard \href{https://en.wikipedia.org/wiki/ISO_8601}{ISO 8601}.
It will help with the visual effect on quickly grasping which month and/or month you are looking at.
It is also incredibly easy to write if you have a numpad on your keyboard.
Also, who does not like a well-sorted list which looks equally properly formatted?? :)

\subsubsection{Tracking Column: Description}
\label{subsec:tracking-column-description}

Put anything sensible that is coherent enough to make you remember, \eg ``John's Supermarket, Smith Street'', ``Coupon for Something, John's 45th birthday, Macy's, Smith Street'' and alike.

\subsubsection{Tracking Column: Amount}
\label{subsec:tracking-column-amount}

Put the amount that is to be attributed to this tracking sheet (and subsequently its category).
In relation to the bits mentioned in \autoref{subsec:tracking-column-description}, these could go into \sheetname{DailyGroceries} and \sheetname{DailyGifts}.
How and why these sheets have these names, all that will be elaborated on in \fullref{subsec:tracking-categories}.

\subsection{Receipts and Documentation}
\label{subsec:receipts}

Collect your receipts, bills, notes and whatever you have that documents the amounts you paid (not: liabilities you will have to pay \emph{in the future}).\footnote{Maybe collect them in a shoebox, an envelope or in a plastic pocket.}
As soon as you worked through a receipt and split up its content into your spending categories, mark it or throw it away instantly.
If you think that you might need the receipt in the future, say for tax purposes or for possibly returning the item, maybe put them in another kind of documentation envelope/folder.

Also, go through your monthly bank statements line by line and check them off.
This most likely leads to delayed entries in the tracking sheets but it ensures factually correct tracking data.

\subsection{Workload}
\label{subsec:workload}

In a way, entering all the data could be interpreted as work.
In my experience, I am content with updating \tfn once a week.
Of course it would still work fine to update it every second, third or fourth week as well.

\subsection{Categories}
\label{subsec:tracking-categories}

A good structure of categories for tracking is the key to a good overlook and analysis of your finances.
I suggest to split up your spending into as many categories as possible.
But again, do what works best \emph{for you}!

For the day-to-day stuff I created the overarching class named \begriff{Daily}.
The Daily class contains multiple categories, such as expenses in regards to bath \& household purchases, food, clothing and miscellaneous items.
Hence I have the tracking sheets \sheetname{DailyBathHousehold} and \sheetname{DailyGroceries} attributed to it.

\subsection{Splitting Receipts}
\label{subsec:splitting-receipts}

When I shopped at a store where I bought multiple kinds of various items, I pick the amounts spent in different categories and write them into the respective tracking sheets for these categories and name the entries accordingly.

\subsection{Tracking: An Example}
\label{subsec:tracking-example}

I spent a total amount of 30.65 EUR at Aldi.
I bought groceries and supplies for the household.
This could be an entry in the sheet for groceries.
\begin{center}\sffamily
	\begin{tabular}{|l|l|r|}
		\multicolumn{3}{l}{Groceries}\\
		\hline
		\textbf{Date} & \textbf{Description} & \textbf{Amount}\rmfamily\\
		\hline
		2018-06-01 & Aldi ABC Street tot. 30.65 & 19.13\\
		\hline
	\end{tabular}
\end{center}
The remainder of \( 30.65 - 19.13 = 11.52 \) euros is to be billed to another category, in this case Household \& Bath Supplies.
So again I go the relevant sheet and enter the data accordingly:
\begin{center}\sffamily
	\begin{tabular}{|l|l|r|}
		\multicolumn{3}{l}{Household \& Bath Supplies}\\			
		\hline
		\textbf{Date} & \textbf{Description} & \textbf{Amount}\\
		\hline
		2018-06-01 & Aldi ABC Street tot. 30.65 & 11.52\\
		\hline
	\end{tabular}
\end{center}

\subsection{Income and Spending over a Certain Time Period}
\label{subsec:income-and-spending-certain-time-period}

The sheet \sheetname{TimePeriod} is for showing you the sums of daily cash outflows and inflows for the given time period.

For that, you need to type out all the names of your tracking sheets in the table below.
Then you can enter the start date and end date for an arbitrary time period and it will show produce the values.

The only requirement is that you did do well on your tracking duties, other than there isn't really a whole lot of magic involved.
To ease the burden on your file and CPU, consider putting \programmcode{No} back in there.
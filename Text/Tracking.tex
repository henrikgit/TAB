\section{Tracking}
\label{sec:tracking}
% !TeX spellcheck = en_US

Tracking is the most riveting part of the guide.
Not really.
Clearly not.
It is probably the most mundane part of this whole thing, but twice as important.
This insanely complex topic can be summed up with: \emph{do the damn tracking}.

And this it is!
You are done.
You now have the broadest understanding about tracking and can get on with it in \tfn.
But perhaps you feel like continuing reading\ldots

\subsection{Categories}
\label{subsec:tracking-categories}

A good structure of categories for tracking is the key to a good overlook and analysis of your finances.
As stated before, do what works best \emph{for you}!
If you want to, you can totally change up the way/pattern by which you name the tracking sheets.

I suggest to split up your spending into as many categories as possible.
For the day-to-day stuff I created the overarching class/field/group\footnote{To this day, I am not 100\,\% sure if either \emph{class}, \emph{field} or group is the better term for this structure item in the hierarchy.} named \sterm{Daily}.
This class contains multiple categories, such as expenses in regards to groceries, bath \& household, food, clothing and miscellaneous items.
Hence I have the tracking sheets \sheetname{DailyGroceries} and \sheetname{DailyBathHousehold} attributed to it, as shown in \autoref{fig:tracking-categories}, \autopageref{fig:tracking-categories}.

\begin{figure}[htp]
	\centering
	\includegraphics{Gfx/Introduction-Tracking-Categories.pdf}
	\caption[Tracking Categories]{These are some of the classes and their categories in Finances.ods.
	They are typical in certain fields of life, ordered by priority and frequency.
	You may create and name your own collection of classes and categories in any way you wish.%
	}
	\label{fig:tracking-categories}
\end{figure}

There are further elaborations to be given:
the classes and their categories are just an excerpt from my own version of \tfn, which is also somewhat incorporated into the file you downloaded.
I have a day-to-day life that the class \sterm{Daily} is for and I am sure its categories are self-explanatory.
The sheet \sheetname{MiscDaily} is a sheet that welcomes all stranger things that do not fit elsewhere.
And it is in fact the sheet that contains all \sterm{miscellaneous} things that do not fit in other classes/categories.
And at work, I need to buy food and drinks as well, so you will find that and other relevant categories in the \sterm{Work} class.

These categories are arguably quite granular and thus plentiful, which is how the structure developed over time, iterated with each edit.
I suppose one can say that I chose a strategy to mirror life in typical areas and split them up.\footnote{The \emph{very first} division of expense categories was inspired by the categories of a spreadsheet by \href{https://www.vertex42.com/}{Vertex42.com}, namely an old version of the Personal Budget Template.}

As stated a few times already, the content in the file is editable.
Hence the fields and their categories shown in \autoref{fig:tracking-categories} are to be understood as examples.
If you wish to do so, you can change it up to make it fit your needs.

For alternative structures of these categories in ``classes'', one could maybe start by putting all monthly bills into the same field \sterm{Bills}, \ie you would end up with sheets like \sheetname{CableTV}, \sheetname{Internet} and \sheetname{InsuranceCar}, etc., regardless of the area in life this particular bill stems from.
And expenses for different kinds of clothing could be grouped up into a class \emph{Clothing}, and thus have sheet names like \sheetname{WorkClothes}, \sheetname{ChildrenClothes} and so forth.
And the same could be done for different kinds of food, as you obviously need to eat, whether you are at home, work or at school.

Or create a structure based on your priorities, ordered by frequency only if you leave out the grouping of ``typicals area in life''.

Granted, some of these are quite silly and not really practical and I am quite sure you already have a good idea of what you want.
It ultimately is up to you.
For the technicalities of adding, removing or editing categories, go to \autoref{subsec:editing-tracking-categories}.

\subsection{What to Track?}
\label{subsec:what-to-track}

You may find yourself wondering on what exactly it is you should and/or could track with \tfn.
Well, as stated, I originally thought to manage my cash.
And this is what the file primarily does.
But besides that, you can of course edit the file to include other kinds of wealth assets as well.

\subsection{How To Enter Tracking Data}
\label{subsec:enter-tracking-data}

There are \sterm{tracking sheets} for entering data on your expenses and income.
These sheets simply contain data which is to be put into 3 columns: date, description and amount.

\subsubsection{Tracking Column: Date}
\label{subsec:tracking-column-date}

Date: Enter the date for the day the expense occurred.
You can do this in two ways:
\begin{enumerate}
	\item Put the date for the day you bought it or ordered it online.
	\item Or--in case you pay with electronic cash, \eg via card--you might put the date the expense's value is actually transferred from your account.
	This might happen in the case of weekends, bank holidays or due to other kinds of service delays for money transfers.
\end{enumerate}
I \emph{strongly} suggest you use the first kind of date, as the date represents the better point in time, \ie the one you planned for.\footnote{This topic will be referenced and elaborated on in \autoref{subsubsec:different-dates-in-real-life}, \autopageref{subsubsec:different-dates-in-real-life}.}
Regardless, you should stay consistent in the method you choose.
The date does get checked for month and day, but not for the year, as I thought to use \tfn for 1 calendar year and have done so ever since.
So be careful that you enter the correct date as that timestamp is used to assign the amount to the right month.
And:
\begin{specialnote}
	Do not lie about the date.
\end{specialnote}

For writing the date into a cell, you may use any format, but I suggest using \codestuff{YYYY-MM-DD} based on the international standard \href{https://en.wikipedia.org/wiki/ISO_8601}{ISO 8601}.
An example for July 4th, 2019: \codestuff{2019-07-04}.
It will help with the visual effect on quickly grasping which month and/or month you are looking at.
It is also incredibly easy to write if you have a numpad on your keyboard.
Also, who does not like a well-sorted list which looks equally properly formatted?? :)

But if you want to change the date format, simply edit the corresponding cell style template to produce the output you wish.
Note: as intended by the mechanism for style templates, this change would affect \emph{all} cells in the entire file which are formatted with this cell style-template.

\subsubsection{Tracking Column: Description}
\label{subsec:tracking-column-description}

Put down anything sensible that is coherent enough to make you remember, \eg ``Supermarket, Smith Street'', ``Gift coupon for John's 45th birthday, Macy's, Smith Street'' and so forth.

\subsubsection{Tracking Column: Amount}
\label{subsec:tracking-column-amount}

Put the amount that is to be attributed to this tracking sheet (and subsequently its category).
In relation to the bits mentioned in \autoref{subsec:tracking-column-description}, these could go into \sheetname{DailyGroceries} and \sheetname{DailyGifts}.
How and why these sheets have these names, all that was explained in \autoref{subsec:tracking-categories}, \autopageref{subsec:tracking-categories}.

If you want to change the format of the figures and have the symbol for your currency displayed, simply edit the corresponding cell style template to produce the output you wish.
Note: as intended by the mechanism for style templates, this change would affect \emph{all} cells in the entire file which are formatted with this cell style-template.

\subsection{Receipts and Documentation}
\label{subsec:tracking-receipts}

Collect your receipts, bills, notes and whatever you have that documents the amounts you paid.\footnote{Maybe put them on/in some kind of holder/retainer thing.}
As soon as you worked through a receipt and split up its content into your spending categories, mark it or throw it away instantly, unless you might need it again.
If you think that you might need the receipt in the future, \eg for tax purposes or for perhaps returning the item, put them in another kind of documentation envelope/folder.\footnote{Maybe collect them in a shoebox, an envelope or in a plastic pocket.}

Also, go through your monthly bank statements line by line and check them off.
This most likely leads to delayed entries in the tracking sheets but it ensures factually correct tracking data.

\subsection{Workload}
\label{subsec:tracking-workload}

In a way, entering all the data could be interpreted as work.
In my experience, I am content with updating \tfn at least every second week.
Of course it would still work fine to update it every second, third or fourth week as well.


\subsection{Splitting Receipts}
\label{subsec:splitting-receipts}

After I shopped at a store where I bought multiple kinds of various items, I pick the amounts spent in different categories and write them into the respective tracking sheets for these categories and name the entries accordingly.

\subsection{Examples for Entering Tracking Data}
\label{subsec:examples-for-entering-tracking-data}

Say you went to Aldi and you spent a total amount of 30.65\,€.
You bought groceries and supplies for the household.
This could be an entry in the sheet for groceries then:
\begin{center}\sffamily
	\begin{tabular}{|l|l|r|}
		\multicolumn{3}{l}{Groceries}\\
		\hline
		\textbf{Date} & \textbf{Description} & \textbf{Amount}\rmfamily\\
		\hline
		2018-06-01 & Aldi, ABC Street tot. 30.65 & 19.13\\
		\hline
	\end{tabular}
\end{center}

The remainder of \( 30.65 - 19.13 = 11.52\) is to be billed to another category, in this case Household \& Bath Supplies.
So again you go the relevant sheet and enter the data accordingly:
\begin{center}\sffamily
	\begin{tabular}{|l|l|r|}
		\multicolumn{3}{l}{Household \& Bath Supplies}\\			
		\hline
		\textbf{Date} & \textbf{Description} & \textbf{Amount}\\
		\hline
		2018-06-01 & Aldi, ABC Street tot. 30.65 & 11.52\\
		\hline
	\end{tabular}
\end{center}

\subsection{Different Values in Finances.ods vs. Real-World Data}
\label{subsec:different-values-finances.ods-vs-real-world}

One basic understanding for working with \tfn is that there is going to be a difference between the amounts which are tracked by \tfn on the one hand, and all the accounts that you decided your use (most definitely the account you use for your salary/wage being one of them).
I regard this as inevitable.
However great and thorough your notes and collection of receipts are, keep in mind that it is only a normal thing to miss things.

You need to decide if a difference in the single digit-range (although I sincerely doubt that this is possible) is acceptable or if you are still ok with low three digits, or perhaps even more.
Before you go ahead and calculate the different, take a look at which items make up the difference:
\begin{itemize}
	\item balance in the accounts your are tracking
	\item the amount of cash in your wallet
	\item the amount of cash that each person with access to the tracked accounts has
\end{itemize}
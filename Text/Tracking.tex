\section{Tracking}
\label{sec:Tracking}

Clearly the most riveting part of the guide.
This insanely complex topic can be summed up with: \emph{do the damn tracking}.

There are \begriff{tracking sheets} for entering data on your expenses and income.
These sheets simply contain data which is to be put into 3 columns:
\begin{itemize}
	\item Date: You may use any format, but I suggest using \programmcode{YYYY-MM-DD} based on the international standard ISO 8601.
	It will help with the visual effect on quickly gr
	\item Description: Put anything in here, \eg Name-of-Supermarkt, Something Street
	\item Amount: The amount
\end{itemize}

The following subsection contain some tips/advice guidance on how to approach the tracking. 

\subsection{Receipts and Documentation}
\label{subsec:receipts}

Collect your receipts, bills, notes and whatever you have that documents the amounts you paid (not: liabilities you will have to pay \emph{in the future}).\footnote{Maybe collect them in a shoebox, an envelope or in a plastic pocket.}
As soon as you worked through a receipt and split up its content into your spending categories, mark it or throw it away instantly.
If you think that you might need the receipt in the future, say for tax purposes or for possibly returning the item, maybe put them in another kind of documentation envelope/folder.

\subsection{Workload}
\label{subsec:workload}

In a way, entering all the data could be interpreted as work.
In my experience, I am content with updating \tfn once a week.
Of course it would still work fine to update it every 2, 3 or 4 weeks as well.

\subsection{Categories}
\label{subsec:tracking-categories}

A good structure of categories for tracking is the key to a good overlook and analysis of your finances.
I suggest to split up your spending into as many categories as possible.
You will find a \todo{hier weitermachen}
But again, do what works \emph{for you}!

\subsection{Splitting Receipts}

When I shopped at a store where I bought multiple kinds of various items, I split the bill, proverbially speaking.
I divide the amounts spent in different categories among the respective tracking sheets for these categories and name the entries accordingly.

An example: I spent a total amount of 30.65 EUR at Aldi.
I bought groceries and supplies for the household.
Then this could be an entry in the sheet for groceries.
\begin{center}\sffamily
	\begin{tabular}{|l|l|r|}
		\multicolumn{3}{l}{Groceries}\\
		\hline
		\textbf{Date} & \textbf{Description} & \textbf{Amount}\rmfamily\\
		\hline
		2018-06-01 & Aldi tot. 30.65 & 19.13\\
		\hline
	\end{tabular}
\end{center}
The remainder of \( 30.65 - 19.13 = 11.52 \) euros is to be billed to another category, in this case Household \& Bath Supplies.
So again you go the relevant sheet and enter the data accordingly:
\begin{center}\sffamily
	\begin{tabular}{|l|l|r|}
		\multicolumn{3}{l}{Household \& Bath Supplies}\\			
		\hline
		\textbf{Date} & \textbf{Description} & \textbf{Amount}\\
		\hline
		2018-06-01 & Aldi tot. 30.65 & 11.52\\
		\hline
	\end{tabular}
\end{center}

% !TeX spellcheck = en_US
\subsection{Thoughts on Budgeting}
\label{subsec:thoughts-on-budgeting}
% !TeX spellcheck = en_US

What about reality, when things go up, down, sideways or in circles?
Life can be a mess and if you really want to use \tfn for tracking and budgeting your cash, I would like to support that by providing some help.

\subsubsection{Thoughts on Non-Monthly Expenses}
\label{subsubsec:thoughts-non-monthly-expenses}

This part builds upon \autoref{subsubsec:bic-monthly-expense}, which was about monthly expenses, but reoccurring expenses that are to be paid at non-monthly intervals (\eg quarterly) were mentioned.
There are further elaborations to be made on the possibilities of non-monthly expenses.
\begin{itemize}
	\item You can set their start and end date to span over the whole year (or 12 months): the expense item would occur after 12 months and it would be the total of all single payments.
	There would be a deviation in all months, but the deviation's total over the whole year equals 0.
	As it is a reoccurring expense, the cell entry for the \ac{bic} sum prohibition would have to be set to \codestuff{Yes}.
	\item For each payment, \ie the expense is regarded as a single purchase and needs the sum prohibition-switch set to \codestuff{No}.
	Hence one would have to create multiple entries for the whole group of quarterly payments.
	There are 2 possibilities one could choose:
	\begin{enumerate}
		\item One could set the start date for all payments to the month of today or whenever one would want to start budget for it.
		Note: this would lead to an increased budgeting load in the ensuing months right after the start date.
		So this would lead to 4 quarterly budget items, all of which start in \codestuff{2019-01}, but end in \codestuff{2019-03}, \codestuff{2019-06}, \codestuff{2019-09} and \codestuff{2019-12}.
		\item One could set the time periods for each quarterly payment to span exactly over each quarter.
		The first payment would be due in march, thus the time period would be \codestuff{2018-01} to \codestuff{2018-03}.
		Accordingly, the next budget item would start in \codestuff{2018-04} and end in \codestuff{2019-06} and so forth.
	\end{enumerate}
\end{itemize}

\subsubsection{Thoughts on Different Dates in Real Life}
\label{subsubsec:thoughts-different-dates}

Dates in regards to planned expenses can differ from how events unfold in reality.
Let us talk about the following questions:
\begin{itemize}
	\item What if when you just want to buy an item \emph{earlier} than the planned month?
	\item What if you somehow end up buying something later than planned?
	\item What if the money transfer goes through too late?
\end{itemize}

To start with the first question, let us look at what happens if you buy something earlier than planned:
\begin{enumerate}
	\item You track the purchase as described in \autoref{sec:tracking}.
	\item You will observe a deviation in the month you purchase the item, as it does not equal the planned month.
	\item The month which you planned the purchase for will generate a surplus.
\end{enumerate}

Well, I suggest that you change its end date to the date you spent the amount.
Why?
First of, the envelope method is used for single purchases only in \tfn.
As this action, \ie purchasing a single item, is a not a monthly expense, the deviation for the month is the one that matters.
Therefor, the timing of the purchase matters.
If you do not change the end date, you would create a deviation in the month you actually purchased it.
Also the positive deviation will reward you in the month it was dated for purchase and that dilutes the structure of your planned expenses unnecessarily.
Therefor I suggest that you change its end date.

But to make sure you are aware of the consequences: as this shortens the time period, the whole amount will be split up over fewer months, therefor this will increase the monthly values for past months.
If your budget overall is arguably tightly calculated, it will possibly add strain to the months that already went by and you could retroactively make them have a budgeted deficit.
Although there is a chance that the opposite might be true as well, that you would not add a critical amount because the deviations in the months prior were in your favor.

About the second question, what if you somehow end up buying something later than planned?
This leads to the same process as above, but simply flipped from the perspective of time.
Overall, if you end up not purchasing something in the planned month, that is on you.
You could move the end date to a later month, however at that point you could ask yourself if the purchase is an expense you actually want to spend.

Also, regardless of your actions or lack thereof, there is the time factor in the money transfer.
The debit could be withdrawn too late or too early from your account and by being the thorough tracker you certainly strive to be, you treat the dates as they occur in real life and ultimately have a unnecessary deviation, but it still happened: you could possibly have overdrawn your checking account or withdrawn an amount too late.
Well, that is life.

All that may be discussed further and deeper, but ultimately you need to decide on the following choices:
\begin{itemize}
	\item Shall budgeting in \tfn reflect real life or should \tfn not be changed and remain set to the planned to data?
	\item Should you wait with the purchase or set a calendar reminder of some sort next time?
\end{itemize}
\begin{specialnote}
	However you decide, never change dates for tracking!
\end{specialnote}

\subsection{Efficient Budgeting and Putting Every Euro to Work}
\label{subsec:efficient-budgeting}

If you manage to spend less than planned in a month, you get a surplus which goes into your following month's budget.
If you go over the budgeted amount with your your expenses in a month, the deficit will give you a decrease of the available funds in the following month.
This is rather basic.

But having long-term deviations bring a rather annoyingly hidden caveat with them.
The surplus for a budget entry gives you ``money back'' in a sense, so that probably provides a good feeling generally.
In case this happens often, the item's budgeted amount is clearly too high as the real-life expenses often are lower than anticipated.
This is counter-productive when you want to optimize your budget allocation.
As your prioritized your expenses, you obviously put a focus on certain items and positioned others in the background.
To undo this, try to rework your budget plan:
\begin{itemize}
	\item Look for problematic budget item.
	\item Shorten its time period (make it stop in the current month) and adjust its amount accordingly so it will have the same budgeted amount per month.
	\item Add a new budget item, which equals the old one but perhaps append its description by ``v2'' or something alike.
	Also, make it start right after the current month.
\end{itemize}
While the budget as a whole gets more and more optimized to achieve the desired goals, you ultimately get to work, for the lack of a better term, with the prioritized expense categories at the right time, get to account for relatively high one-time/emergency expenses and generate an optimized allocation of funds.

The deficit does the opposite of the surplus and will most likely cause a bad notion at least.
But the long-term effect is, in the case the tracking sheets in \tfn are true and contain numbers based in reality, that you will unhinge your budget plan and cause financial ruin.

\subsubsection{Thoughts on Isolated Deviations for Categories}
\label{subsubsec:thoughts-on-isolated-deviations-for-categories}

Deviation is a core element of budgeting.\footnote{See \autoref{subsubsec:deviation}, \autopageref{subsubsec:deviation}.}
In general, \emph{all} budget categories get summed and lumped together each month.
The argument could be made that the deviation of a budget category or budget class should be calculated just for itself, but that would be dismissive in regards to the bigger picture as your cash is the main asset which gets managed, not some budget category.
Also, the next month could very well not have any amount budgeted, hence the surplus or deficit would just linger around and/or could be carried over until a month in which a new budgeted expense occurs, which would be arguably nonsensical.
Hence the file uses the approach that you use put all available funds in one pot and then budget them.

\subsubsection{Thoughts on Delayed Budgeting}
\label{subsubsec:thoughts-delayed-budgeting}

Say a situation similar to this one occurs:
it is the last day of the month and you would go on a spontaneous shopping trip.
You know you still have some unused, yet unspecified, amount to spend.
So you go ahead and spend that amount.
The next time you open your own version of \tfn the next time, it is the next month and you enter the tracking correctly.
You will then see a deviation because you did not account for that expense in your budget plan.

The nice way of going on with this is what I would call \sterm{delayed budgeting}.
However, better put, this is simply an expense which might or might not be covered by your budget plan and thus will lead to a deviation or not.
You should enter the tracking data it occurred and deal with the deviation.
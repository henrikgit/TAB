\subsection{Thoughts on Budgeting}
\label{subsec:thoughts-on-budgeting}
% !TeX spellcheck = en_US

What about reality, when things go up, down, sideways or in circles?
Life can be a mess and if you really want to use \tfn for tracking and budgeting your cash, I would like to support that by providing some help in the form of thoughts.

\subsubsection{Thoughts on Non-Monthly Expenses}
\label{subsubsec:thoughts-non-monthly-expenses}

This part is in regards to \autoref{subsubsec:budgeting-item-column-monthly-expense}, where reoccurring expenses that are to be paid at non-monthly intervals (\eg quarterly) were mentioned.
There are further elaborations to be made on the possibilities of non-monthly expenses.
\begin{itemize}
	\item You can set their start and end date to span over the whole year (or 12 months): the expense item would occur after 12 months and it would be the total of all single payments.
	There would be a deviation in all months, but the yearly total deviation equals 0.
	Obviously, as it is a reoccurring expense, the switch for \begriff{sum prohibition} would have to be set to \programmcode{Yes}.
	\item For each payment, \ie the expense is regarded as a single purchase and needs the \begriff{sum prohibition} switch set to \programmcode{No}.
	Hence one would have to create multiple entries for the whole group of quarterly payments.
	There are 2 possibilities one could choose:
	\begin{enumerate}
		\item One could set the start date for all payments to the month of today or whenever one would want to start budget for it.
		Note: this would lead to an increased budgeting load in the ensuing months right after the start date.
		So this would lead to 4 quarterly budget items, all of which start in \programmcode{2019-01}, but end in \programmcode{2019-03}, \programmcode{2019-06}, \programmcode{2019-09} and \programmcode{2019-12}.
		\item One could set the start date to the month after a payment happened.
		Explanation: let us assume the first payment would be due in march, thus the time period would be \programmcode{2018-01} to \programmcode{2018-03}.
		Accordingly, the next budget item could start in \programmcode{2018-04} and end it \programmcode{2019-06}.
	\end{enumerate}
\end{itemize}

\subsubsection{Thoughts on Different Dates in Real Life}
\label{subsubsec:thoughts-different-dates}

Dates in regards to planned expenses can differ from how events unfold in reality.
Firstly, what if when you just want to buy an item \emph{earlier} than the planned month?
Then the following process happens:
\begin{enumerate}
	\item You track the purchase as described in \autoref{sec:tracking}.
	\item You will observe a deviation in the month you purchase the item, as it does not equal the planned month.
\end{enumerate}
I suggest that you backdate its end date.

First of, the envelope method is used for single purchases only.
Otherwise the budget will penalize you in the month it was dated for purchase and that dilutes the structure of your planned expenses unnecessarily.
As this action, \ie purchasing a single item, is a not a monthly expense, the deviation for the month is the one that matters.
Therefor, the timing of the purchase matters and if you do not backdate the end, you will have created an unnecessary deviation in the month you actually purchase it.

But this will increase the monthly values for past months, and if your budget overall is arguably tight, it will possibly add strain to the months that already went by.
Also, you would be adding an amount of money to the already budgeted amount in the past months, which would be kind of fake as all kinds of expenses have been spent, all kinds of budget items were planned more or less carefully so that it all made sense from a financial perspective.
So that the past months might not be diluted by the fake 

Secondly, what if you somehow end up buying something later than planned?
Well, if you end up not purchasing something in the right month, that is on you.
This leads to the same \todo{weiter}

But that is ok and you could be happy about the positive effect on the deviation.

As mentioned \autoref{subsec:tracking-column-date}, there are 2 ways (at least the ones I listed) that dates can be used:
\begin{enumerate}
	\item A date for the day something was bought/ordered.
	Regardless of any kind of money transfer, this is the date the purchase was planned for.
	\item With regards to any kind of money transfer, there is also date--in case one uses some form of electronic payment (\eg a card)--on which the amount is actually transferred from your account.
\end{enumerate}

Finally, regardless of your actions or lack thereof, there is the time factor in the money transfer.
The debit could be withdrawn too late or too early from your account and by being the thorough tracker you certainly strive to be, you treat the dates as they occur in real life and ultimately have a unnecessary deviation, but it still happened: you could possibly have overdrawn your checking account or withdrawn an amount too late.
Well, that is life.
Regardless, you update your list to reflect your view. (because who does not like a well-maintained list, right?). \todo{punkt checken wegen einmal plan-datum und real datum}

All that may be discussed further and deeper, but ultimately you need to decide on the following choices:
\begin{itemize}
	\item Shall budgeting in \tfn reflect real life or should \tfn not be changed and remain set to the planned to data?
	\item Should you wait with the purchase or set a calendar reminder of some sort next time?
\end{itemize}
\begin{specialnote}
	But however you decide, never change dates for tracking.
\end{specialnote}

\subsubsection{Thoughts on Delayed Budgeting}
\label{subsubsec:thoughts-delayed-budgeting}

Say one the following situation occurs:
\begin{itemize}
	\item It is the last day of the month.
	\item One would go on a spontaneous shopping trip.
	\item You know you have some certain, yet unspecified, amount to spend.
	\item One would spend that amount.
	\item When you open your version of \tfn the next time, it is the next month and you will see a deviation
\end{itemize}

The nice way of going on with this is what I would call \begriff{delayed budgeting}.
But this is not a \todo{fehlt}

\subsubsection{Thoughts on Deviations for Categories}
\label{subsubsec:thoughts-deviations-for-categories}

Ausführungen\todo{fehlt}
\begin{itemize}
	\item In general, the budget categories get summed and lumped together each month.
	\item One could argue that the budget categories should be 
	\item Dealing with (delayed) deviations can result in a double-edged sword:
	\begin{itemize}
		\item On one hand, it feels kind of ``good'' to have a surplus for a budget item.\\
		\ldots in case this happens often, this item is clearly budgeted
		\item On the other hand, 
		\item \ldots wrong.
	\end{itemize}
\end{itemize}
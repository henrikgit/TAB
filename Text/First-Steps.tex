\section{First Steps}
\label{sec:first-steps}

\subsection{Introducing LibreOffice Calc}
\label{subsec:introducing-libreoffice-calc}

LibreOffice is a free office suite and great software.\footnote{See \href{https://www.libreoffice.org/discover/libreoffice/}{https://www.libreoffice.org/discover/libreoffice/}.}
You can download it on its   \href{https://www.libreoffice.org/download/download/}{official download page} of course.

Per \href{https://www.libreoffice.org/discover/calc/}{its official website}:
\begin{quote}\small
	Calc is the free spreadsheet program you've always needed. Newcomers find it intuitive and easy to learn, while professional data miners and number crunchers appreciate the comprehensive range of advanced functions. Built-in wizards guide you through choosing and using a comprehensive range of advanced features. Or you can download templates from the LibreOffice template repository, for ready-made spreadsheet solutions.
\end{quote}

\subsubsection{Help and Tips for LibreOffice Calc}
\label{subsubsec:help-tips-for-libreoffice-calc}

Providing help for working with \loc and writing a full introduction about it is clearly beyond the scope of this document.
I will, though, give thoughts on some key aspects I regard as crucial, but before that there are these important links and documents for proper help on \loc:
\begin{itemize}
	\item The most important bit beforehand: \loc is not so different from Excel.
	Many things you might have picked up while working with Excel are probably going to work in a similar--if not identical--fashion in Calc.
	\item The Official Documentation is really great.\footnote{URL: \href{https://documentation.libreoffice.org/en/english-documentation/}{https://documentation.libreoffice.org/en/english-documentation/}.}
	But even better is the \begriff{Getting Started} guide, which was released for LibreOffice 6.\footnote{URL: \href{https://documentation.libreoffice.org/en/english-documentation/getting-started-guide/}{https://documentation.libreoffice.org/en/english-documentation/getting-started-guide/}.}
	Please start with pages 1--14, which are important overall, but especially for Mac users.
	The actual good stuff is the section for Calc, which starts on page 118.
	\item By the way, if you are using Windows and Adobe Reader for reading PDF-files, you can quickly jump to any page with the keyboard shortcut \keystroke{Ctrl}+\keystroke{N}.
	In Evince on Linux, use \keystroke{Ctrl}+\keystroke{L}.
	\item For the sake of thoroughness, it is worth mentioned that you please not ever hesitate to use \href{http://www.duckduckgo.com}{http://www.duckduckgo.com} and  \href{https://www.google.com}{Google}.
\end{itemize}

\subsubsection{First Steps with LibreOffice Calc}
\label{subsubsec:first-steps-with-libreoffice-calc}

Here are some suggestions for first steps if you intend to follow through with learning how to work with \loc.
Again, you might be able to utilize knowledge you gained while working with Excel or Google Sheets.

To work with \tfn in a proper and productive way, you should definitely learn these keyboard shortcuts:
\begin{itemize}
	\item Copying content is done, as usual, with \keystroke{Ctrl}+\keystroke{C}.
	\item However, pasting in Calc can be done in multiple ways:
	\begin{enumerate}
		\item The well-known way: \keystroke{Ctrl}+\keystroke{V}.
		This keeps the content in cache and you can paste it again.
		\item The other way is simply pressing \keystroke{\( \hookleftarrow \)} (Enter).
		This does not keep the content in cache.
		\item The best way: \keystroke{Ctrl}+\keystroke{Shift}+\keystroke{V}.
		If you copied the content in a spreadsheet file, Calc will show a dialog window named \begriff{Paste Special}.
		In this you can select the intended attributes of the paste action while simultaneously performing operations.
		With hitting \keystroke{\( \hookleftarrow \)} or clicking on \keystroke{Ok}, the copied content gets pasted to the currently marked cell.
		The content gets kept in cache and you can paste it again.
	\end{enumerate}
	\item Deleting content, format-wise and content-wise, is done via \keystroke{\( \longleftarrow \)} (Backspace).
	This will leave a blank cell.
	\item For the case of copying/replacing cells in bulk, look at keyboard shortcuts on page 129 in \begriff{Getting Started}.
	For example, if a cell in a bigger group/block of cells is selected and you would like to additionally select all the cells to the right of the current cell in this block, use the shortcut \keystroke{Ctrl}+\keystroke{Shift}+\keystroke{\( \rightarrow \)}.
\end{itemize}

But should also definitely learn about \begriff{cell style templates} and the \begriff{navigator}:
\begin{itemize}
	\item Cell style templates are the core properties of formatting the whole thing.
	Without them, one would go crazy when editing \tfn (ok, you got me, I am talking about \emph{me}).
	The templates can be found in the stylist (press \keystroke{F11} for that), which enable you to make fast changes to the look (format-wise) of the file.
	Note: if you change one of them, you are changing the formatting of all cells in the entire document that have been formatted with this style template.
	And this effect is the intended behavior.
	\item The navigator (reachable via \keystroke{F5}) enables you to navigate arguably fast between the sheets if you do not use keyboard shortcuts.
	It not only lists all visible sheets in the file, but also all other kinds of elements that can be found in it.
\end{itemize}

\subsection{Synchronization}
\label{subsec:synchronization}

Simply save the file in a folder that belongs to the folder which get synchronized with personal cloud software, \eg Dropbox.

\subsection{Security}
\label{subsec:security}

For securing the file, use the password-protection when you save it.
For more guidance on that, please consult the \begriff{Getting Started} guide, page 30.

\begin{specialnote}
Do not lose the password!
\end{specialnote}

\subsection{The Start}
\label{subsec:opening-the-file}

As you first open \tfn, you should navigate to the sheet \sheetname{2019Start}.\todo{weiter}

sheet called \begriff{Introduction} is the first one you'll find at the beginning of the file.
If not, please navigate to it.
For this, please either open the navigator by pressing \keystroke{F5} (this is a toggle operation, pressing \keystroke{F5} again will close it) or go to the task bar an click on \programmcode{View} \( \blacktriangleright \) \programmcode{Navigator}.

As its name declares, it features an overview of several things:
\begin{itemize}
	\item First of, on top there are the the budget headers for the month before the current month, the current months and the two consecutive ones.
\end{itemize}

% !TeX spellcheck = en_US
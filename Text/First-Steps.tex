\section{First Steps}
\label{sec:First-Steps}

\subsection{Introducing LibreOffice Calc}
\label{subsec:libreoffice-calc}

LibreOffice is a free office suite and great software.\footnote{See \href{https://www.libreoffice.org/discover/libreoffice/}{https://www.libreoffice.org/discover/libreoffice/}.}
You can download it on the  \href{https://www.libreoffice.org/download/download/}{official download page} of course.

Per \href{https://www.libreoffice.org/discover/calc/}{its official website}:
\begin{quote}\small
	Calc is the free spreadsheet program you've always needed. Newcomers find it intuitive and easy to learn, while professional data miners and number crunchers appreciate the comprehensive range of advanced functions. Built-in wizards guide you through choosing and using a comprehensive range of advanced features. Or you can download templates from the LibreOffice template repository, for ready-made spreadsheet solutions.
\end{quote}

Now before I elaborate on some key aspects I regard as crucial, here are the most important links for finding proper help on \loc:
\begin{itemize}
	\item \href{https://documentation.libreoffice.org/en/english-documentation/}{The Official Documentation} is really great.
	\item But even better is the \begriff{Getting Started} guide, which was released for LibreOffice 6.
	Please start with pages 1--14, which are important overall, but especially for Mac users.
	The actual good stuff is the section for Calc, which starts on page 118.
	By the way, you can quickly jump to that page with the shortcut \keystroke{Ctrl}+\keystroke{N} in Adobe Reader.
	In Evince, use \keystroke{Ctrl}+\keystroke{L}.
	\item For the sake of thoroughness, here are 2 links for some up-and-coming search machine: \href{http://www.google.com}{Google} and the awesome \href{https://www.duckduckgo.com}{DuckDuckGo}.
\end{itemize}

Here are some suggestions for first steps if you intend to follow through with learning how to work with \loc.
The most important bit beforehand: you can relax a bit. \loc is not so different from Excel.
To properly work with \tfn, you should learn about \begriff{cell templates} and the \begriff{navigator}.
\begin{itemize}
	\item Cell style templates are the core properties of formatting the whole thing.
	Without them, one would have go crazy when editing \tfn.
	The templates can be found in the stylist (press \keystroke{F5} for that), which enable you to make fast changes to the look (format-wise) of the file.
	\item The navigator enables you to navigate arguably fast between the sheets if you do not use keyboard shortcuts.
	It not only lists all visible sheets in the file, but also all other kinds of elements that can be found in it.
	\item For the case of editing formulas and/or copying/replacing cells in bulk, look at keyboard shortcuts on page 129 in \begriff{Getting Started}.
	They are arguably identical to the ones you might be familiar with in Excel.
	For example, if you would like select the block of cells to the right of the current cell, use the shortcut \keystroke{Shift} + \keystroke{Ctrl} + \keystroke{\( \rightarrow \)}.
	Then just hit \keystroke{Enter}.
\end{itemize}

\subsubsection{Synchronization}
\label{subsubsec:synchronization}

Simply save the file in a folder that belongs to the folder which get synchronized with Dropbox(-like) software.

\subsubsection{Security}
\label{subsubsec:security}

For securing the file, use the password-protection when you save it.
For more guidance on that, please consult the \begriff{Getting Started} guide, page 30.
Do not lose the password because if you do, you cannot open it again!

\subsection{The Start}
\label{subsec:opening-the-file}

As you first open \tfn, you should navigate 

sheet called \begriff{Introduction} is the first one you'll find at the beginning of the file.
If not, please navigate to it.
For this, please either open the navigator by pressing \tbox{F5} (this is a toggle operation, pressing \tbox{F5} again will close it) or click on \programmcode{View} above in the task bar and then on \programmcode{Navigator}.

As its name declares, it features an overview of several things:
\begin{itemize}
	\item First of, on top there are the the budget headers for the month before the current month, the current months and the two consecutive ones.
\end{itemize}

% !TeX spellcheck = en_US
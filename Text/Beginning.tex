% !TeX spellcheck = en_US

\begin{center}\LARGE
Track and Budget
\end{center}
\begin{center}\scshape\large\sffamily
	---the guide---
\end{center}

\vfill
\begin{center}
Version: \today, \thistime
\end{center}

\clearpage
%~\\
%\vfill
%{\sffamily\centering Foreword\label{foreword}\\}
%\small\singlespacing\rmfamily
\begin{center}\large\scshape
	Foreword
\end{center}
First, to clarify, \tfn is not a file for tracking and creating a budget in the strict sense of the word.
If you rework your planned spending on a monthly basis, literature (and many search results online which you may find if you search for ``budgeting vs. forecasting'') frames it as a tool which shall serve rather to budget and to forecast.

In this guide, I try to convey knowledge via tips, explanations and ``rules''.
Please regard all that as a collection of \emph{suggestions}--do what works best for you!

Since \tfn is a free file, you are obviously going to create your own copy of it so you might as well get into the details and change up a few things.
If you do so, I would very much like to receive some feedback of course. :)
More on that in \fullref{sec:feedback}.

And you will undoubtedly develop your own sense of working with \tfn, what is best to be used in a certain way or another, what you could edit or what you would like to leave alone.
Do not worry if you think it is a bit much at first glance, you are most definitely going to get a better understanding in time.

%Finally, something technical should be noted: if you read this file on your computer/laptop or so, you will see colored boxes.
Finally, something technical should be noted: if you read this file on your computer/laptop or another screen, you will see colored text.
They indicate hyperlinks and can be clicked.
\begin{itemize}
	%	\item Gray boxes are links to targets inside this very document.
	%	\item Blue-ish boxes are links to website and will be opened in your browser.
	\item Text formatted in dark red is a link to a target inside this very document.
	\item Blue-ish text bits are links to website and will be opened in your browser.
\end{itemize}
%They do \emph{not} get printed out and are just an overlay in your PDF viewer.
%Again, the boxes do not get printed.
If you clicked on a link to jump to another target inside this document, you can go back to the old location by clicking on the big left arrow-button or using the keyboard shortcut \keystroke{Alt}+\keystroke{\( \leftarrow \)}  in Adobe Reader.
In Evince, use \keystroke{Alt}+\keystroke{P}.

\section{Roadmap}
\label{sec:roadmap}

\emph{Roadmap} might be a bit of an overstatement.
At this stage, something like \begriff{loose thoughts for future features and enhancements} is be a more appropriate phrase.
You can find the 3 most prominent roadmap items in \autoref{subsec:translations} to \autoref{subsec:excel}.
Obviously, everything that is mentioned contingent on the demand and support for them.

\subsection{Translations}
\label{subsec:translations}

Translating \tfn is incredibly time-consuming, as every single word has to be manipulated.
It is especially tiresome in regards to manipulating formulas and their (text) results.

I would certainly welcome any contribution! :)

\subsection{Additional Features or Changes}
\label{subsec:additional-features}

Whether integrating additional features in entirely new sheets or in an existing one or changing something in regards to optimization, I welcome it.
This would be great and--as outlined in \autoref{subsec:motivation-history}--is one of the key factors for putting this online.

\subsection{Excel}
\label{subsec:excel}

First of, the short explanation which can be summed up with 2 bullet points:
\begin{itemize}
	\item I personally do not own Excel (in its current version).
	\item Hence I am arguably pessimistic about simply saving \tfn as a \programmcode{.xlsx}-file and then distributing it.
\end{itemize}

The longer explanation:
I am aware of Excel being the de facto standard in the industry.
In terms of productivity software, its reach is possibly unparalleled.
Although I am arguably very proficient in Excel and sometimes even write super cool macros at work, I personally do not own Excel in its current version (whatever exactly that is nowadays).

I have reached a point where I am happy with Calc on its own.
I see no reason why I should buy Excel.
Ever since I created \tfn, I of course had doubts if it was the right thing to do it all in \loc.
Well, yes, it was.
The catch is: personally, I prefer using free software, especially when I manage to reach the same goal with zero or negligible differences in effort.
Therefor, at this time, I do not think it is feasible for me to somehow port the file to Excel as it might be more complex than just saving the file as \programmcode{.xlsx}.\footnote{I would not be able to proof-check it and perform the necessary (style) edits in Excel afterwards.
But if you want to, go ahead and I would greatly appreciate any feedback! :)}

If I were to own a current Excel version at some point, one of the first porting steps would definitely be to use macros to automate lots of tasks and then publish \tfn as an Excel file as well.
One thing I can say is that I am (slowly) learning Python which is also free software.
When I become proficient enough, I am going to use it for the aforementioned automation tasks.

\subsection{Miscellaneous Roadmap Items}
\label{subsec:misc-roadmap-items}

At this point, some other items on the agenda are:
\begin{itemize}
	\item Currently the budgeting amount of an item is divided up evenly (\ie linear) over the budgeting period.
	The amount could be spread out in a non-linear fashion, \eg progressively or degressively.
	\item Some kind of function to create a backup\ldots and then to re-import the old tracking sheets?
	\item Create a website for this thing.
	\item World domination.
\end{itemize}

% !TeX spellcheck = en_US
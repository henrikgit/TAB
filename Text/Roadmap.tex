\section{Roadmap}
\label{sec:roadmap}

\emph{Roadmap} might be a strong word.
At this stage, something like \begriff{loose thoughts for future features and enhancements} might be a more appropriate phrase.
Obviously, everything about the roadmap is contingent on the demand and support for them.
You can find the biggest 3 thoughts below.

\subsection{Translations}
\label{subsec:translations}

Translating \tfn is incredibly time-consuming, as every single word has to be manipulated.
It is especially tiresome in regards to manipulating formulas and their (text) results.

I would certainly welcome any contribution! :)

\subsection{Additional Features or Changes}
\label{subsec:additional-features}

Whether integrating additional features in entirely new sheets or in an existing one or changing something in regards to optimization, I welcome it.
This would be great and--as outlined in \fullref{subsec:motivation-history}--is one of the key factors for putting this online.

\subsection{Excel}
\label{subsec:excel}

I am aware of Excel being the de facto standard in the industry.
In terms of productivity software, its reach is possibly unparalleled.
Although I am arguably highly proficient in Excel and am very capable of writing macros (which I do at work sometimes), I personally do not own Excel in its current version (whatever exactly that is nowadays).

However I have reached a point where I am happy with Calc alone.
I see no reason why I should buy Excel.
Ever since I created \tfn, I of course had doubts if it was the right thing to do it all in \loc.
Well, yes, it was.
The catch is: personally, I prefer playing around in free software, especially when I manage to reach the same goal with zero or negligible differences in effort.
Therefor, at this time, I do not think it is feasible for me to somehow port the file to Excel as it might be more complex than just saving the file as \programmcode{.xlsx}.\footnote{But if you want to, go ahead and please give me feedback!}

If I were to own a current Excel version at some point, one of the first porting steps would definitely be to use macros to automate lots of kinds of tasks and then publish \tfn as an Excel file as well.
One thing I can say is that I am (slowly) learning Python.
When I become proficient enough, I am going to use it for the aforementioned automation tasks.

% !TeX spellcheck = en_US